\begin{deluxetable}{lcccc}[htp]
\tablecaption{Upper Limits on the Rate of Optical Flashes in SNe\,Ia \label{tab:flash}}
\tablehead{
\colhead{}
& \colhead{}
& \colhead{}
& \multicolumn{2}{c}{Flash Fraction} \\
\cline{4-5}
\colhead{$M_\mathrm{disc}$}
& \colhead{~~~$N_\mathrm{SN}$~~~}
& \colhead{~~~$N_\mathrm{flash}$~~~}
& \colhead{~~~C\&P~~~}
& \colhead{~~~Jeffreys~~~}
} 
\startdata
$> -16.5$\,mag & 33 & 0 & $<0.11$ & $<0.07$ \\
$> -16.0$\,mag & 15 & 0 & $<0.22$ & $<0.15$ \\
$> -15.5$\,mag & 8 & 0 & $<0.37$ & $<0.26$ \\
\enddata
\tablecomments{$N_\mathrm{SN}$ is the number of SNe with an absolute magnitude
at the epoch of discovery ($M_\mathrm{disc}$) fainter than the given cuts
($-16.5$, $-16.0$, $-15.5$\,mag) in both the \gztf\ and \rztf\ bands.
$N_\mathrm{flash}$ is the number of SNe with observed flashes. The flash
fraction represents the 95\% confidence interval upper limit on the rate of
early flashes from SNe\,Ia. It has been calculated two ways: (i) using the
method of \citet{Clopper34}, and (ii) using Jeffreys prior \citep[see][]{Cai05}.
}
\end{deluxetable}