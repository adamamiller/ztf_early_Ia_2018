
%% Beginning of file 'sample63.tex'
%%
%% Modified 2019 June
%%
%% This is a sample manuscript marked up using the
%% AASTeX v6.3 LaTeX 2e macros.
%%
%% AASTeX is now based on Alexey Vikhlinin's emulateapj.cls 
%% (Copyright 2000-2015).  See the classfile for details.

%% AASTeX requires revtex4-1.cls (http://publish.aps.org/revtex4/) and
%% other external packages (latexsym, graphicx, amssymb, longtable, and epsf).
%% All of these external packages should already be present in the modern TeX 
%% distributions.  If not they can also be obtained at www.ctan.org.

%% The first piece of markup in an AASTeX v6.x document is the \documentclass
%% command. LaTeX will ignore any data that comes before this command. The 
%% documentclass can take an optional argument to modify the output style.
%% The command below calls the preprint style which will produce a tightly 
%% typeset, one-column, single-spaced document.  It is the default and thus
%% does not need to be explicitly stated.
%%
%%
%% using aastex version 6.3
\documentclass[twocolumn]{aastex63}

\usepackage{amsmath}
\usepackage{empheq}
\usepackage{mathrsfs}
\usepackage{textcomp}
\usepackage{enumitem}   
\usepackage{gensymb}
\usepackage{hyperref}
\usepackage{graphicx}
\usepackage[caption=false]{subfig}
\usepackage{multirow}
\usepackage{longtable}
\hypersetup{colorlinks, linkcolor={blue}, citecolor={blue}, urlcolor={blue}} 

\usepackage{lineno}
% \linenumbers
%% The default is a single spaced, 10 point font, single spaced article.
%% There are 5 other style options available via an optional argument. They
%% can be invoked like this:
%%
%% \documentclass[arguments]{aastex63}
%% 
%% where the layout options are:
%%
%%  twocolumn   : two text columns, 10 point font, single spaced article.
%%                This is the most compact and represent the final published
%%                derived PDF copy of the accepted manuscript from the publisher
%%  manuscript  : one text column, 12 point font, double spaced article.
%%  preprint    : one text column, 12 point font, single spaced article.  
%%  preprint2   : two text columns, 12 point font, single spaced article.
%%  modern      : a stylish, single text column, 12 point font, article with
%% 		  wider left and right margins. This uses the Daniel
%% 		  Foreman-Mackey and David Hogg design.
%%  RNAAS       : Preferred style for Research Notes which are by design 
%%                lacking an abstract and brief. DO NOT use \begin{abstract}
%%                and \end{abstract} with this style.
%%
%% Note that you can submit to the AAS Journals in any of these 6 styles.
%%
%% There are other optional arguments one can invoke to allow other stylistic
%% actions. The available options are:
%%
%%   astrosymb    : Loads Astrosymb font and define \astrocommands. 
%%   tighten      : Makes baselineskip slightly smaller, only works with 
%%                  the twocolumn substyle.
%%   times        : uses times font instead of the default
%%   linenumbers  : turn on lineno package.
%%   trackchanges : required to see the revision mark up and print its output
%%   longauthor   : Do not use the more compressed footnote style (default) for 
%%                  the author/collaboration/affiliations. Instead print all
%%                  affiliation information after each name. Creates a much 
%%                  longer author list but may be desirable for short 
%%                  author papers.
%% twocolappendix : make 2 column appendix.
%%   anonymous    : Do not show the authors, affiliations and acknowledgments 
%%                  for dual anonymous review.
%%
%% these can be used in any combination, e.g.
%%
%% \documentclass[twocolumn,linenumbers,trackchanges]{aastex63}
%%
%% AASTeX v6.* now includes \hyperref support. While we have built in specific
%% defaults into the classfile you can manually override them with the
%% \hypersetup command. For example,
%%
%% \hypersetup{linkcolor=red,citecolor=green,filecolor=cyan,urlcolor=magenta}
%%
%% will change the color of the internal links to red, the links to the
%% bibliography to green, the file links to cyan, and the external links to
%% magenta. Additional information on \hyperref options can be found here:
%% https://www.tug.org/applications/hyperref/manual.html#x1-40003
%%
%% Note that in v6.3 "bookmarks" has been changed to "true" in hyperref
%% to improve the accessibility of the compiled pdf file.
%%
%% If you want to create your own macros, you can do so
%% using \newcommand. Your macros should appear before
%% the \begin{document} command.
%%
\newcommand{\vdag}{(v)^\dagger}
\newcommand\aastex{AAS\TeX}
\newcommand\latex{La\TeX}

%%%%%%%%%%%%%%%%%%%%%%%%%%%%%%%%%%%%%%%%%%%%%%%%%%%%%%%%%%%%%%%%%%%%%%%%%%%%%%%%
%%
%% The following section defines new commands for comments from co-authors
%%
\definecolor{DarkOrange}{RGB}{204, 85, 0}
\definecolor{LincolnGreen}{RGB}{17, 102, 0}
\def\ion#1#2{#1$\;${\footnotesize\rm{#2}}\relax}

\newcommand{\yy}[1]{{\color{red} yy: {#1}}}
\newcommand{\aam}[1]{{\color{DarkOrange} aam: {#1}}}
\newcommand{\todo}[1]{{\color{magenta} to-do: {#1}}}

\newcommand{\rztf}{$r_\mathrm{ZTF}$}
\newcommand{\gztf}{$g_\mathrm{ZTF}$}
\newcommand{\tfl}{$t_\mathrm{fl}$}
\newcommand{\trise}{$t_\mathrm{rise}$}
\newcommand{\tbmax}{$T_{B,\mathrm{max}}$}
%%
%%%%%%%%%%%%%%%%%%%%%%%%%%%%%%%%%%%%%%%%%%%%%%%%%%%%%%%%%%%%%%%%%%%%%%%%%%%%%%%%

%% Reintroduced the \received and \accepted commands from AASTeX v5.2
\received{\today}
% \revised{January 10, 2019}
% \accepted{\today}
%% Command to document which AAS Journal the manuscript was submitted to.
%% Adds "Submitted to " the argument.
\submitjournal{ApJ}

%% For manuscript that include authors in collaborations, AASTeX v6.3
%% builds on the \collaboration command to allow greater freedom to 
%% keep the traditional author+affiliation information but only show
%% subsets. The \collaboration command now must appear AFTER the group
%% of authors in the collaboration and it takes TWO arguments. The last
%% is still the collaboration identifier. The text given in this
%% argument is what will be shown in the manuscript. The first argument
%% is the number of author above the \collaboration command to show with
%% the collaboration text. If there are authors that are not part of any
%% collaboration the \nocollaboration command is used. This command takes
%% one argument which is also the number of authors above to show. A
%% dashed line is shown to indicate no collaboration. This example manuscript
%% shows how these commands work to display specific set of authors 
%% on the front page.
%%
%% For manuscript without any need to use \collaboration the 
%% \AuthorCollaborationLimit command from v6.2 can still be used to 
%% show a subset of authors.
%
%\AuthorCollaborationLimit=2
%
%% will only show Schwarz & Muench on the front page of the manuscript
%% (assuming the \collaboration and \nocollaboration commands are
%% commented out).
%%
%% Note that all of the author will be shown in the published article.
%% This feature is meant to be used prior to acceptance to make the
%% front end of a long author article more manageable. Please do not use
%% this functionality for manuscripts with less than 20 authors. Conversely,
%% please do use this when the number of authors exceeds 40.
%%
%% Use \allauthors at the manuscript end to show the full author list.
%% This command should only be used with \AuthorCollaborationLimit is used.

%% The following command can be used to set the latex table counters.  It
%% is needed in this document because it uses a mix of latex tabular and
%% AASTeX deluxetables.  In general it should not be needed.
%\setcounter{table}{1}

%%%%%%%%%%%%%%%%%%%%%%%%%%%%%%%%%%%%%%%%%%%%%%%%%%%%%%%%%%%%%%%%%%%%%%%%%%%%%%%%
%%
%% The following section outlines numerous optional output that
%% can be displayed in the front matter or as running meta-data.
%%
%% If you wish, you may supply running head information, although
%% this information may be modified by the editorial offices.
\shorttitle{The Rise of ZTF SNe Ia}
\shortauthors{Miller et al.}
%%
%% You can add a light gray and diagonal water-mark to the first page 
%% with this command:
\watermark{DRAFT}
%% where "text", e.g. DRAFT, is the text to appear.  If the text is 
%% long you can control the water-mark size with:
%% \setwatermarkfontsize{dimension}
%% where dimension is any recognized LaTeX dimension, e.g. pt, in, etc.
%%
%%%%%%%%%%%%%%%%%%%%%%%%%%%%%%%%%%%%%%%%%%%%%%%%%%%%%%%%%%%%%%%%%%%%%%%%%%%%%%%%

%% This is the end of the preamble.  Indicate the beginning of the
%% manuscript itself with \begin{document}.

\begin{document}

\title{ZTF Early Observations of Type Ia  Supernovae II: \\ First Light, the Initial Rise, and Time to Reach Maximum Brightness}

%% LaTeX will automatically break titles if they run longer than
%% one line. However, you may use \\ to force a line break if
%% you desire. In v6.3 you can include a footnote in the title.

%% A significant change from earlier AASTEX versions is in the structure for 
%% calling author and affiliations. The change was necessary to implement 
%% auto-indexing of affiliations which prior was a manual process that could 
%% easily be tedious in large author manuscripts.
%%
%% The \author command is the same as before except it now takes an optional
%% argument which is the 16 digit ORCID. The syntax is:
%% \author[xxxx-xxxx-xxxx-xxxx]{Author Name}
%%
%% This will hyperlink the author name to the author's ORCID page. Note that
%% during compilation, LaTeX will do some limited checking of the format of
%% the ID to make sure it is valid. If the "orcid-ID.pdf" image file is 
%% present or in the LaTeX pathway, the OrcID icon will appear next to
%% the authors name.
%%
%% Use \affiliation for affiliation information. The old \affil is now aliased
%% to \affiliation. AASTeX v6.3 will automatically index these in the header.
%% When a duplicate is found its index will be the same as its previous entry.
%%
%% Note that \altaffilmark and \altaffiltext have been removed and thus 
%% can not be used to document secondary affiliations. If they are used latex
%% will issue a specific error message and quit. Please use multiple 
%% \affiliation calls for to document more than one affiliation.
%%
%% The new \altaffiliation can be used to indicate some secondary information
%% such as fellowships. This command produces a non-numeric footnote that is
%% set away from the numeric \affiliation footnotes.  NOTE that if an
%% \altaffiliation command is used it must come BEFORE the \affiliation call,
%% right after the \author command, in order to place the footnotes in
%% the proper location.
%%
%% Use \email to set provide email addresses. Each \email will appear on its
%% own line so you can put multiple email address in one \email call. A new
%% \correspondingauthor command is available in V6.3 to identify the
%% corresponding author of the manuscript. It is the author's responsibility
%% to make sure this name is also in the author list.
%%
%% While authors can be grouped inside the same \author and \affiliation
%% commands it is better to have a single author for each. This allows for
%% one to exploit all the new benefits and should make book-keeping easier.
%%
%% If done correctly the peer review system will be able to
%% automatically put the author and affiliation information from the manuscript
%% and save the corresponding author the trouble of entering it by hand.

% \correspondingauthor{Adam A. Miller}


\author[0000-0001-9515-478X]{A.~A.~Miller}
\affiliation{Center for Interdisciplinary Exploration and Research in Astrophysics and Department of Physics and Astronomy, Northwestern University, 1800 Sherman Ave, Evanston, IL 60201, USA}
\affiliation{The Adler Planetarium, Chicago, IL 60605, USA}
\email{amiller@northwestern.edu}

\author[0000-0001-6747-8509]{Y.~Yao}
\affiliation{Cahill Center for Astrophysics, 
             California Institute of Technology, 
             1200 E.~California Boulevard, Pasadena, CA 91125, USA}

\author[0000-0002-8255-5127]{M.~Bulla} 
\affiliation{Nordita, KTH Royal Institute of Technology and Stockholm
University, Roslagstullsbacken 23, SE-106 91 Stockholm, Sweden}

\author[0000-0002-1128-3662]{C.~Pankow}
\affiliation{Center for Interdisciplinary Exploration and Research in Astrophysics (CIERA) and Department of Physics and Astronomy, Northwestern University, 2145 Sheridan Road, Evanston, IL 60208, USA}

\author[0000-0001-8018-5348]{E.~C.~Bellm}
\affiliation{DIRAC Institute, Department of Astronomy, University of Washington, 3910 15th Avenue NE, Seattle, WA 98195, USA}

\author[0000-0003-1673-970X]{S.~B.~Cenko}
\affiliation{Astrophysics Science Division, NASA Goddard Space Flight Center, 8800 Greenbelt Road, Greenbelt, MD 20771, USA}
\affiliation{Joint Space-Science Institute, University of Maryland, College Park, MD 20742, USA}

\author{R.~Dekany}
\affiliation{Caltech Optical Observatories, California Institute of Technology, Pasadena, CA 91125, USA}
\author[0000-0002-4223-103X]{C.~Fremling}
\affiliation{Cahill Center for Astrophysics, 
             California Institute of Technology, 
             1200 E.~California Boulevard, Pasadena, CA 91125, USA}

\author[0000-0002-3168-0139]{M.~J.~Graham}
\affiliation{Division of Physics, Mathematics, and Astronomy,
             California Institute of Technology, Pasadena, CA
             91125, USA}

\author[0000-0002-6540-1484]{T.~Kupfer}
\affiliation{Kavli Institute for Theoretical Physics, University of California, Santa Barbara, CA 93106, USA}
\author[0000-0003-2451-5482]{R.~R.~Laher}
\affiliation{IPAC, California Institute of Technology, 1200 E.~California
             Blvd, Pasadena, CA 91125, USA}


\author[0000-0003-2242-0244]{A.~A.~Mahabal}
\affiliation{Division of Physics, Mathematics, and Astronomy,
             California Institute of Technology, Pasadena, CA
             91125, USA}
\affiliation{Center for Data Driven Discovery, California Institute of Technology, Pasadena, CA 91125, USA}

\author[0000-0002-8532-9395]{F.~J.~Masci}
\affiliation{IPAC, California Institute of Technology, 1200 E.~California
             Blvd, Pasadena, CA 91125, USA}

\author[0000-0002-3389-0586]{P.~E.~Nugent}
\affiliation{Computational Cosmology Center, Lawrence Berkeley National Laboratory, 1 Cyclotron Road, Berkeley, CA 94720, USA}
\affiliation{Department of Astronomy, University of California, Berkeley, CA 94720-3411, USA}

\author[0000-0002-0387-370X]{R.~Riddle}
\affiliation{Caltech Optical Observatories, California Institute of Technology, Pasadena, CA 91125, USA}

\author[0000-0001-7648-4142]{B.~Rusholme}
\affiliation{IPAC, California Institute of Technology, 1200 E.~California
             Blvd, Pasadena, CA 91125, USA}

\author[0000-0001-7062-9726]{R.~M.~Smith}
\affiliation{Caltech Optical Observatories, California Institute of Technology, Pasadena, CA 91125, USA}

\author[0000-0003-4401-0430]{D.~L.~Shupe}
\affiliation{IPAC, California Institute of Technology, 1200 E.~California
             Blvd, Pasadena, CA 91125, USA}

\author[0000-0002-2626-2872]{J.~van Roestel}
\affiliation{Division of Physics, Mathematics, and Astronomy,
             California Institute of Technology, Pasadena, CA
             91125, USA}

\author[0000-0001-5390-8563]{S.~R.~Kulkarni}
\affiliation{Cahill Center for Astrophysics, 
             California Institute of Technology, 
             1200 E.~California Boulevard, Pasadena, CA 91125, USA}

%% Note that the \and command from previous versions of AASTeX is now
%% depreciated in this version as it is no longer necessary. AASTeX 
%% automatically takes care of all commas and "and"s between authors names.

%% AASTeX 6.3 has the new \collaboration and \nocollaboration commands to
%% provide the collaboration status of a group of authors. These commands 
%% can be used either before or after the list of corresponding authors. The
%% argument for \collaboration is the collaboration identifier. Authors are
%% encouraged to surround collaboration identifiers with ()s. The 
%% \nocollaboration command takes no argument and exists to indicate that
%% the nearby authors are not part of surrounding collaborations.

%% Mark off the abstract in the ``abstract'' environment. 
\begin{abstract}

While it is clear that Type Ia supernovae (SNe) are the result of
thermonuclear explosions in C/O white dwarfs (WDs), a great deal remains
uncertain about the binary companion that facilitates the explosive disruption
of the WD. Here, we present a comprehensive analysis of a unique, and large,
data set of 127 SNe Ia with exquisite coverage by the Zwicky Transient
Facility (ZTF). High-cadence (6 observations per night) ZTF observations allow
us to measure the SN rise time and examine its initial evolution. We develop a
Bayesian framework to model the early rise as a power-law in time, which
enables the inclusion of priors in our model. For a volume-limited subset of
normal SNe Ia, we find the mean power-law index is consistent with 2 in the
\rztf-band ($\alpha_r = 2.01\pm0.02$), as expected in the expanding fireball
model. There are, however, individual SNe that are clearly inconsistent with
$\alpha_r=2$. We estimate a mean rise time of \edit1{18.9\,d} (with a range
extending from $\sim$15--22\,d), though this is subject to the adopted prior.
We identify an important, previously unknown, bias whereby the rise times for
higher redshift SNe within a flux-limited survey are systematically
underestimated. This effect can be partially alleviated if the power-law index
is fixed to $\alpha=2$, in which case we estimate a mean rise time of
\edit1{21.7\,d} (with a range from $\sim$18--23\,d). The sample includes a
handful or rare and peculiar SNe Ia. Finally, we conclude with a discussion of
lessons learned from the ZTF sample that can eventually be applied to Large
Synoptic Survey Telescope observations. \todo{Gonzalez-Gaitan find long rise
times for fast decliners, and short rise times for slow decliners}

% In line with previous studies we find that super-Chandrasekhar explosions and
% SNe Ia interacting with their circumstellar medium have longer rise times than
% normal SNe Ia. We also measure the rise time of ZTF18abclfee (SN\,2018cxk), a
% 2002cx-like/Iax SN, to be $10.01 \pm^{0.4}_{0.33}$\,d. This is the most
% precise rise time estimate for an 02cx-like SN to date.

\end{abstract}

%% Keywords should appear after the \end{abstract} command. 
%% See the online documentation for the full list of available subject
%% keywords and the rules for their use.
\keywords{supernovae: general --- methods: observational --- surveys --- catalogs --- supernovae: individual (ZTF18abclfee/SN\,2018cxk)}

%% From the front matter, we move on to the body of the paper.
%% Sections are demarcated by \section and \subsection, respectively.
%% Observe the use of the LaTeX \label
%% command after the \subsection to give a symbolic KEY to the
%% subsection for cross-referencing in a \ref command.
%% You can use LaTeX's \ref and \label commands to keep track of
%% cross-references to sections, equations, tables, and figures.
%% That way, if you change the order of any elements, LaTeX will
%% automatically renumber them.
%%
%% We recommend that authors also use the natbib \citep
%% and \citet commands to identify citations.  The citations are
%% tied to the reference list via symbolic KEYs. The KEY corresponds
%% to the KEY in the \bibitem in the reference list below. 

\vspace{1em}

\section{Introduction}

The fact that supernovae (SNe) of Type Ia can be empirically calibrated as
standardizable candles makes them arguably the most important tool in all of
physics for the past $\sim$two decades. By unlocking our ability to accurately
measure distances at high redshift, SNe Ia have revolutionized our
understanding of the Universe \citep{Riess98,Perlmutter99}. While it is all
but certain that SNe Ia are the result of thermonuclear explosions in
carbon-oxygen (C/O) white dwarfs (WDs) in binary star systems (see
\citealt{Maoz14,Livio18}), there remains a great deal about SNe Ia progenitors
and the precise explosion mechanism that we do not know. This leads to the
tantalizing hope that an improved understanding of the binary companions or
explosion may improve our ability to calibrate these standardizable candles.

One clear avenue for better understanding the progenitors of SNe Ia is to
obtain observations in the hours to days after explosion \citep{Maoz14}. Such
detections provide an opportunity to probe the progenitor environment and
binary companion, which is simply not possible once the SN evolves well into
the expansion phase (they are standardizable precisely because they are all
nearly identical at maximum light). Indeed, in the landmark discovery of
SN\,2011fe, \citet{Nugent11} were able to constrain the time of explosion to
$\pm 20$\,min (though see \citealt{Piro13,Piro14} for an explanation of a
potential early ``dark phase''). \citet{Bloom12a} would later combine the
observations in \citet{Nugent11} with an early non-detection while comparing
the limits to shock-breakout models to constrain the size of the progenitor to
be $\lesssim 0.01\,R_\odot$, providing the most direct evidence to date that
SNe Ia come from WDs. These critical findings have demonstrated the importance
of high-cadence time-domain surveys for discovering SNe shortly after
explosion.

Pinning down the binary companion to the exploding WD remains a challenge.
There are likely two dominant pathways towards explosion. In the first, the WD
accretes H of He from, or merges with the core of, a non-degenerate companion
and eventually explodes as it approaches the Chandrasekhar mass
($M_\mathrm{Ch}$; known as the single degenerate, or SD, scenario;
\citealt{Whelan73}), while in the second the explosion follows the interaction
or merger of two WD stars (known as the double degenerate, or DD, scenario;
\citealt{Webbink84}). While the debate long focused on which of these two
scenarios is correct, there is now empirical evidence in support of both
channels. PTF\,11kx, an extreme example of a SN Ia, showed evidence of
multiple shells of circumstellar material \citep{Dilday12}, which is precisely
what one would expect in a WD$+$red giant system that had undergone multiple
novae prior to the final, fatal thermonuclear explosion (see \citealt{Soker13}
for an alternative explanation for PTF\,11kx). On the other hand,
hypervelocity WDs discovered by \textit{Gaia} are likely the surviving
companions of DD explosions \citep{Shen18}.

There is also emerging evidence that WDs can explode prior to reaching
$M_\mathrm{Ch}$. Detailed modeling of SNe Ia light curves \citep{Scalzo14a}
and a blue-to-red-to-blue color evolution observed in a few SNe
\citep{Jiang17,Noebauer17,Polin19,De19}, point to sub-$M_\mathrm{Ch}$ mass
explosions. Such explosions are possible if a C/O WD accretes and retains a
thick He shell. A detonation in this shell can trigger an explosion in the C/O
core of the WD (e.g., \citealt{Nomoto82,Nomoto82a}).

Now the most pressing questions are the following: which binary companion WD
(DD) or non-degenerate star (SD), and which mass explosion, $M_\mathrm{Ch}$ or
sub-$M_\mathrm{Ch}$, is dominant?

Here too, early observations should prove useful. In the SD scenario the SN
ejecta will collide with the non-degenerate companion creating a shock that
gives rise to a bright ultraviolet/optical flash in the days after explosion
\citep{Kasen10a}. To date, the search for such a signature in large samples
typically results in a non-detection (e.g.,
\citealt{Hayden10,Ganeshalingam11,Bianco11}).\footnote{There are claims of
companion interaction based on short-lived optical ``bumps'' in the early
light curves of individual SNe (e.g.,
\citealt{Cao15,Marion16,Hosseinzadeh17,Dimitriadis19}). Alternative models
(e.g., \citealt{Dessart14,Piro16,Levanon17,Magee20a}) utilizing different
physical scenarios can produce similar bumps, leading many (e.g.,
\citealt{Kromer16,Noebauer17,Miller18,Shappee18,Shappee19}) to appeal to
alternative explanations than ejecta-companion interaction.} For the DD
scenario, some sub-$M_\mathrm{Ch}$ DD explosions exhibit a highly unusual
color evolution, though this can only be observed a few days after explosion
\citep{Noebauer17,Polin19}.

Measurements of the rise time, i.e., the time it takes the SN to evolve from
first light, the moment when optical photons diffuse out of the photosphere,
to maximum brightness, can also play a role in constraining the progenitor
systems of SNe Ia. Initial work to estimate the rise times of SNe Ia clearly
demonstrated that early efforts to model WD explosions significantly
underestimated the opacities in the SN ejecta (e.g., \citealt{Riess99a}).
Furthermore, while the famous luminosity-decline relationship for SNe Ia makes
them standardizable \citep{Phillips93}, recent evidence suggests that the
rise, rather than the decline, of SNe Ia is a better indicator of their peak
luminosity \citep{Hayden19}. Rise time measurements are most precise when a SN
is discovered shortly after explosion, which only becomes routine with
high-cadence observations.

In their seminal study, \citet{Riess99a} found that the mean rise time of SNe
Ia is $19.5 \pm 0.2$\,d, \textit{after correcting the individual SNe for the
luminosity-decline relation} (we hereafter refer to these corrections as shape
corrections). Follow-up studies estimated a similar mean rise time for
high-redshift SNe Ia \citep{Aldering00,Conley06}.\footnote{\citet{Aldering00}
importantly point out that rise time estimates can be significantly biased if
uncertainties in the time of maximum light are ignored.} In \citet{Hayden10}
and \citet{Ganeshalingam11}, similar approaches were applied to significantly
larger samples of SNe, and shorter shape-corrected mean rise times were found.

As observational cadences have increased over the past $\sim$decade, there has
been a surge of SNe Ia discovered shortly after explosion. This has led more
recent efforts to focus on measuring the rise times of populations of
individual SNe (e.g., \citealt{Firth15,Zheng17a,Papadogiannakis19}), which is
the approach adopted in this study. The utility of avoiding shape corrections
is that it allows one to search for multiple populations in the distribution
of rise times, which could point to a multitude of explosion scenarios. While
\citet{Papadogiannakis19} found no evidence for multiple populations,
\citet{Ganeshalingam10} found that high-velocity SNe Ia rise $\sim$1.5\,d
faster than their normal counterparts.

We are now in an era where hydrodynamic radiation transport models have become
very sophisticated. Accurate measurements of the observed distribution of SN
Ia rise times can be compared with these models to rule out theoretical
scenarios that evolve too quickly or too slowly (e.g., \citealt{Magee18}). In
a similar sense, every model produces unique predictions for the initial
evolution of the SN (e.g., \citealt{Dessart14,Noebauer17,Polin19,Magee20}).
Similarly, if the early emission is modeled as a power-law in time (i.e., $f
\propto t^\alpha$), measures of the power-law index $\alpha$ can confirm, or
reject, different explosion/ejecta-mixing scenarios \citet{Magee20}.

In this paper, the second in a series of three examining the photometric
evolution of 127 SNe Ia with early observations discovered by the Zwicky
Transient Facility (ZTF; \citealt{Bellm19,Graham19,Dekany20}) in 2018, we
examine the rise time of SNe Ia and whether or not their early emission can be
characterized as a simple power law. Paper I \citep{Yao19} describes the
sample, while Paper III \citep{Bulla20} discusses the evolution of SNe Ia
colors shortly after explosion. The sample, which is large, is equally
impressive in its quality: ZTF observations are obtained in both the \gztf\
and \rztf\ filters every night. These criteria, 1\,d cadence and multiple
filters, are essential to constrain the distribution of $^{56}$Ni in the SN
ejecta (e.g., \citealt{Magee20}). The nightly cadence additionally allows us
to include sub-threshold detections in our analysis of SNe Ia \citep{Yao19}.
This aspect of the ZTF sample separates it from other low-$z$ data sets. We
construct a Bayesian framework to estimate the rise time of individual SNe. An
advantage of this approach, relative to other studies, is that it allows us to
naturally incorporate priors into the model fitting. We uncover a systematic
bias whereby the rise times of the higher redshift SNe within a flux-limited
survey are typically underestimated. We show that the adoption of strong
priors can, at least partially, alleviate this bias. Finally, we conclude with
a discussion of lessons from ZTF that can be applied to the Vera C.~Rubin
Observatory \edit1{Legacy} Survey of Space and Time (LSST).

\section{ZTF Photometry}\label{sec:ztf}

The sample of 127 SNe Ia utilized in this study was defined in \citet{Yao19},
we refer the reader there for full details on the sample selection. Briefly,
the SNe studied herein were observed as part of the high-cadence extragalactic
experiment conducted by the ZTF partnership in 2018 \citep{Bellm19a}. This
experiment monitors $\sim$3000\,$\deg^2$ on a nightly basis (over the 9 month
period when the fields are visible), with the aim of obtaining 3 \gztf\ and 3
\rztf\ observations every night. In total, there were 247 spectroscopically
confirmed SNe Ia discovered within these fields. The GROWTH Marshal
\citep{Kasliwal19} is used to organize and visualize the data. Following cuts
to limit the sample to SNe that were discovered ``early'' (defined as 10\,d or
more, in the SN rest frame, prior to the time of maximum light in the
$B$-band, \tbmax) and have high quality light curves, the sample was reduced
to 127 SNe (see \citealt{Yao19} for the full details).

In \citet{Yao19}, we produced ``forced'' point-spread-function (PSF)
photometry for each of the 127 SNe on every image covering the position of the
SN. The PSF model was generated as part of the ZTF real-time image subtraction
pipeline \citep{Masci19}, which uses an image-differencing technique based on
\citet{Zackay16}. The forced PSF photometry procedure fixes the position of
each SN and measures the PSF flux in all images that contain the SN position,
even in epochs where the signal-to-noise ratio (SNR)\,$\la 1$ and the SN is
not detected.

We normalize the SN flux relative to the observed peak flux in the \gztf- and
\rztf-bands as measured by \texttt{SALT2} (\citealt{Guy07}; see
\citealt{Yao19} for our \texttt{SALT2} implementation details). The relative
fluxes produced via this procedure are unique for every ZTF reference image
(hereafter fcqf\,ID following the nomenclature in \citealt{Yao19}). The ZTF
field grid includes some overlap, and SNe that occur in overlap regions will
have multiple fcqf\,IDs for a single filter. Estimates of the baseline flux
must account for the individual fcqf\,IDs. In \citet{Yao19}, this baseline
flux, $C$, and $\chi^2_{\nu}$, which accounts for underestimated uncertainties
in the flux measurements, are used to correct the results of the forced PSF
photometry. For this study, we do not employ the corrections suggested in
\citet{Yao19} and instead incorporate these values into our model so they can
be marginalized over and effectively ignored.

\section{A Search for Early Optical Flashes}\label{sec:flash}

\edit1{We constrain the presence of extreme optical flashes in the initial
phase of SNe\,Ia by searching our sample for sources with light curves that
initially decline before following the typical rise of an SN\,Ia. While this
simple criterion excludes events that exhibit an early bump, such as
SNe\,2017cbv and 2018oh, ZTF has found SNe\,Ia with an early optical flash
\citep[e.g., SN\,2019yvq;][]{Miller20a}. Furthermore, the detection of bumps,
as opposed to ``flashes'' where the flux is observed to decline, requires
physical models of the early emission \citep[e.g.,][]{Levanon17}, and cannot
be captured by the empirical power-law models used in this study. A detailed
search for ``bumps'' in the early light curves of this sample will be
presented in a future study (M.~Deckers et al., in preparation).}

\edit1{Of the 127 SNe in our sample, only two show a decline in flux following
the epoch of discovery. When accounting for the uncertainties on the
individual flux measurements, however, the observed decline in both
ZTF18abklljv (SN\,2018lpk) and ZTF18abptsco (AT\,2018lpm) is statistically
consistent with a constant flux during the first two nights of detection. We
therefore conclude that an early optical flash is not detected in any of the
SNe in our sample.}

\edit1{While we have not detected an early flash in any of the SNe in our
sample, we can still constrain the rate of such events using binonmial
statistics. With no detections in 127 SNe, a naive estimate of the rate of
optical flashes in SNe\,Ia is $\lesssim$3\%. However, not every SN in our
sample is discovered sufficiently early to rule out the presence of a flash.
Using the distance modulii from \citet{Yao19} and the host-galaxy reddening
and $K$-corrections from \citet{Bulla20}, we split our sample based on the
absolute magnitude at the epoch of discovery, and use these subsets to
constrain the rate of optical flashes in SNe\,Ia as summarized in
Table~\ref{tab:flash}.} 

\begin{deluxetable}{lcccc}[htp]
\tablecaption{Upper Limits on the Rate of Optical Flashes in SNe\,Ia \label{tab:flash}}
\tablehead{
\colhead{}
& \colhead{}
& \colhead{}
& \multicolumn{2}{c}{Flash Fraction} \\
\cline{4-5}
\colhead{$M_\mathrm{disc}$}
& \colhead{~~~$N_\mathrm{SN}$~~~}
& \colhead{~~~$N_\mathrm{flash}$~~~}
& \colhead{~~~C\&P~~~}
& \colhead{~~~Jeffreys~~~}
} 
\startdata
$> -16.5$\,mag & 33 & 0 & $<0.11$ & $<0.07$ \\
$> -16.0$\,mag & 15 & 0 & $<0.22$ & $<0.15$ \\
$> -15.5$\,mag & 8 & 0 & $<0.37$ & $<0.26$ \\
\enddata
\tablecomments{$N_\mathrm{SN}$ is the number of SNe with an absolute magnitude
at the epoch of discovery ($M_\mathrm{disc}$) fainter than the given cuts
($-16.5$, $-16.0$, $-15.5$\,mag) in both the \gztf\ and \rztf\ bands.
$N_\mathrm{flash}$ is the number of SNe with observed flashes. The flash
fraction represents the 95\% confidence interval upper limit on the rate of
early flashes from SNe\,Ia. It has been calculated two ways: (i) using the
method of \citet{Clopper34}, and (ii) using Jeffreys prior \citep[see][]{Cai05}.
}
\end{deluxetable}

\edit1{With only 8 SNe fainter than $M = -15.5$\,mag at the epoch of
discovery, we find a limit on faint optical flashes of $\lesssim$30\%. Flashes
this faint are expected when the SN ejecta collide with main-sequence
companions \citep[e.g.,][]{Kasen10a}. This limit is consistent with previous
estimates \citep[e.g.,][]{Bianco11}, though not that constraining given that
the companion-interaction signature is only expected in $\sim$10\% of SNe
\citep{Kasen10a}.}


\section{Modeling the Early Rise of SNe Ia}\label{sec:model}

Following arguments first laid out in \citet{Riess99a}, the rest-frame optical
flux of a SN Ia should evolve $\propto t^2$ shortly after explosion. For an
ideal, expanding fireball, the observed flux through a passband covering the
Rayleigh-Jeans tail of the approximately blackbody emission will be $f \propto
R^2 T = v^2 t^2 T$, where $f$ is the SN flux, $T$ is the blackbody
temperature, $R$ is photospheric radius, $v$ is the ejecta velocity, and $t$
is the time since explosion.\footnote{All times reported in this study have
been corrected to the SN rest frame.} While these idealized conditions are not
perfectly met in nature, $T$ and $v$ clearly change shortly after explosion
(e.g., \citealt{Parrent12}), their relative change is small compared to $t$.
Thus, the ``$t^2$-law'' should approximately hold, and indeed many studies of
large samples of SNe Ia have shown that in the blue optical filters $f \propto
t^2$ to within the uncertainties (e.g., \citealt{Conley06, Hayden10,
Ganeshalingam11}). At the same time, several recent studies of individual,
low-redshift SNe Ia show strong evidence that a power-law model for the SN
flux only reproduces the data if the power-law index, $\alpha$, is
significantly lower than 2 (e.g.,
\citealt{Zheng13,Zheng14,Shappee16,Miller18,Fausnaugh19,Dimitriadis19}).

For this study, we characterize the early emission in a single filter from a
SN Ia as a power law:
%
\begin{equation}
    f_b(t) = C + H[t_\mathrm{fl}] A_b (t - t_\mathrm{fl})^{\alpha_b},
    \label{eqn:flux_model}
\end{equation}
%
where $f_b(t)$ is the flux in filter $b$ as a function of time $t$ in the SN
rest frame, $C$ is a constant representing the baseline flux present in the
reference image prior to the SN, $t_\mathrm{fl}$ is the time of first
light,\footnote{$t_\mathrm{fl}$ is not the time of explosion, but rather the
time when optical emission begins for the SN, as the observed emission due to
radioactive decay in the interior of the SN ejecta must first diffuse through
the photosphere, see e.g., \citet{Piro13,Piro14}.} $H[t_\mathrm{fl}]$ is the
Heaviside function equal to 0 for $t < t_\mathrm{fl}$ and 1 for $t \ge
t_\mathrm{fl}$, $A_b$ is a constant of proportionality in filter $b$, and
$\alpha_b$ is the power-law index describing the rise in filter $b$.

For ZTF, observations are obtained in the \gztf- and \rztf-bands, and, we
model the evolution in both filters simultaneously. While, strictly speaking,
$t_{\mathrm{fl}, g} \ne t_{\mathrm{fl}, r}$, we expect these values to be
nearly identical given the similarity of the SN ejecta opacity at these
wavelengths (e.g., Figure~6 in \citealt{Magee18}), and assume we cannot
resolve the difference with ZTF data. Therefore we adopt \tfl\ ($\approx
t_{\mathrm{fl}, g} \approx t_{\mathrm{fl}, r}$) as a single parameter for our
analysis. As discussed in \citet{Yao19}, $C$ is a function of fcqf\,ID, which
represents both the filter and ZTF field ID (see also \S\ref{sec:ztf}).

Many of the SNe in our sample are observed at an extremely early phase in
their evolution, at times when the spectral diversity in SNe Ia is not well
constrained (see for example the bottom panel of Figure~1 in \citealt{Guy07}).
As a result, we do not apply $K$-corrections to the ZTF light curves prior to
model fitting. Furthermore, without precise knowledge of the time of
explosion, it is impossible to know which observations in the ZTF nightly
sequence should be corrected and which should not. While examining SN\,2011fe,
\citet{Firth15} find that ignoring $K$-corrections leads to a systematic
uncertainty on $\alpha$ of $\pm0.1$, which is smaller than the typical
uncertainty we measure (see \S\ref{sec:mean_parameters}). This suggests that
our inability to apply $K$-corrections does not significantly affect our final
conclusions.

If we assume that the observed deviations between the model flux and the data
are the result of Gaussian scatter, the log-likelihood for the data is:
%
\begin{equation}
    \ln \mathscr{L} \propto -\frac{1}{2}\sum_{d,i} \frac{[f_{d,i} - f_d(t_i)]^2}{(\beta_d \sigma_{d,i})^2} -\sum{\ln (\beta_d \sigma_{d,i})},
\end{equation}
%
where the sum is over all fcqf\,IDs $d$ and all observations $i$. $f_{d,i}$
is the $i^\mathrm{th}$ flux measurement with corresponding uncertainty
$\sigma_{d,i}$, and $\beta_d$ is a term we add to account for the fact that
the uncertainties are underestimated (see \citealt{Yao19}). Finally,
$f_d(t_i)$ is the model, Equation~\ref{eqn:flux_model} evaluated at the time
of each observation $t_i$, with $C$ replaced by $C_d$, the baseline for the
individual fcqf\,IDs, and $A_b$ and $\alpha_b$ replaced by $A_{b\mid d}$ and
$\alpha_{b\mid d}$, respectively, as these terms depend on fcqf\,ID, but
only the filter $b$ and not the field ID.

Ultimately, we only care about 3 model parameters: \tfl, and the power-law
index describing the rise in the \gztf\ and \rztf\ filters, hereafter
$\alpha_g$ and $\alpha_r$, respectively. Following Bayes' Law, we multiply the
likelihood by a prior and use an affine-invariant, ensemble Markov chain Monte
Carlo (MCMC) technique \citep{Goodman10} to approximate the model posterior.

There is a strong degeneracy between $A_{b\mid d}$ and $\alpha_{b\mid d}$,
which we find can be removed with the following change of variables
$A^\prime_{b\mid d} = A_{b\mid d} 10^{\alpha_{b\mid d}}$ in
Equation~\ref{eqn:flux_model}. We adopt Jeffreys prior \citep{Jeffreys46} for
the scale parameters $A_{b\mid d}$ and $\beta_d$, and wide flat priors for all
other model parameters, as summarized in Table~\ref{tab:priors}. The MCMC
integration is performed using \texttt{emcee} \citep{Foreman-Mackey13}. Within
the ensemble, we use 100 walkers, each of which is run until convergence or 3
million steps, whichever comes first. We test for convergence by examining the
average autocorrelation length of the individual chains $\tau$ after every
20,000 steps, and consider the chains converged if $n_\mathrm{steps} > 100
\,\tau$, where $n_\mathrm{steps}$ is the total number of steps in each chain,
and the change in $\tau$ relative to the previous estimate has changed by
$<1\%$.

\begin{deluxetable}{llc}[htp]
\tablecaption{Model Parameters $\theta$ and Their Priors \label{tab:priors}}
\tablehead{
\colhead{$\theta$}
& \colhead{Description}
&\colhead{Prior}
} 
\startdata
$C_d$ & baseline flux per fcqfID $d$ & $\mathcal{U}(-10^8,10^8)$ \\
$t_\mathrm{fl}$ & time of first light & $\mathcal{U}(-100,0)$ \\
$A^\prime_{b\mid d}$ & proportionality factor per filter $b$ & ${A^\prime_{b\mid d}}^{-1} 10^{-\alpha_{b\mid d}}$ \\
$\alpha_{b\mid d}$ & rising power-law index per filter $b$ & $\mathcal{U}(0,10^8)$ \\
$\beta_{d}$ & uncertainty scale factor per fcqfID $d$ & $\beta_{d}^{-1}$ \\
\enddata
\tablecomments{The factor of $10^{-\alpha_{b\mid d}}$ in the prior for $A^\prime_{b\mid d}$ follows from the change of variables.
}
\end{deluxetable}

A key decision in modeling the early evolution of SNe Ia light curves is
deciding what is meant by ``early.'' While the simplistic power-law models
adopted here and elsewhere can describe the flux of SNe Ia shortly after
explosion, it is obvious that these models cannot explain the full evolution
of SNe Ia as they never turn over and decay. Throughout the literature there
are various definitions of early, ranging from some studies defining early
relative to the amount of time that has passed following the epoch of
discovery (e.g., \citealt{Nugent11,Zheng13,Miller18}), to others defining it
relative to the time of $B$-band maximum (e.g.,
\citealt{Riess99a,Aldering00,Conley06,Dimitriadis19}), while others define
early in terms of the fractional flux relative to maximum light (e.g.,
\citealt{Olling15,Firth15,Fausnaugh19}). Here we adopt the latter definition
to be consistent with recent work using extremely high-cadence, high-precision
light curves from the space-based \textit{Kepler} K2 \citep{Howell14} and the
Transiting Exoplanet Survey Satellite (\textit{TESS}; \citealt{Ricker15})
missions (e.g., \citealt{Olling15,Fausnaugh19}). As in \citet{Olling15}, we
only include observations up to 40\% of the peak amplitude of the
SN.\footnote{We do this separately in the \gztf\ and \rztf\ filters. In
practice, we subtract a preliminary estimate of the flux baseline derived from
the median flux value for all observations that occurred $>20$\,d (in the SN
rest frame) before \tbmax. We then divide all flux values by the peak flux
determined in \citet{Yao19}. Finally, we calculate the inverse-variance
weighted mean flux for every night of observations, and only retain those
nights with $f_\mathrm{mean} \le 0.4 f_\mathrm{max}$ for model fitting.} We
find that this particular choice, 40\% instead of 30\% or 50\%, does slightly
affect the final inference for the model parameters (see for
\ref{sec:flux_cut} further details).

Of the 127 SNe Ia in our sample, we find that the MCMC chains converge for
every SNe but one, ZTF18aaqnrum (SN\,2018bhs). Nevertheless, we retain it in
our sample as $n_\mathrm{steps} \approx 81 \,\tau$ after 3 million steps,
suggesting several independent samples within the chains (this SN is later
excluded from the sample, see \ref{sec:qa}). From the MCMC chains we can
derive constraints on \tfl, $\alpha_g$, and $\alpha_r$. We show example corner
plots illustrating good, typical, and poor constraints on the model parameters
in Figures~\ref{fig:corner_good}, \ref{fig:corner_median}, and
\ref{fig:corner_bad}, respectively. In this context good, typical, and poor
are defined relative to the width of the 90\% credible region for \tfl\
($\mathrm{CR}_{90}$). Roughly, the good models have $\mathrm{CR}_{90} \la
1.5$\,d (approximately 34 SNe), the median models have $\mathrm{CR}_{90}
\approx 2.5$\,d ($\sim$61 SNe), and the poor models have $\mathrm{CR}_{90} \ga
4$\,d ($\sim$32 SNe). From the corner plots it is clear that there is a
positive correlation between $\alpha_g$ and $\alpha_r$, which makes sense
given the relatively similar regions of the spectral energy distribution (SED)
traced by these filters. Finally, \tfl\ exhibits significant covariance with
each of the $\alpha$ parameters. While we report marginalized credible regions
on all model parameters in Table~\ref{tab:uninformative}, the full posterior
samples should be used for any analysis utilizing the results of our model
fitting (see e.g., \citealt{Bulla20}).

\begin{figure*}
    \centering
    \includegraphics[width=5.2in]{./figures/Fig1.pdf}
    %
    \caption{\textit{Top}: Corner plot showing the posterior constraints on
    \tfl, $\alpha_g$, $\alpha_r$, and the respective constants of
    proportionality, $A_g^\prime$ and $A_r^\prime$ for ZTF18abgmcmv
    (SN\,2018eay). ZTF18abgmcmv is a SN that is well fit by the model. For
    clarity, the $C_d$ and $\beta_d$ terms are excluded (in general they do
    not exhibit strong covariance with the parameters shown here as they are
    tightly constrained by the pre-SN observations). Marginalized
    one-dimensional distributions are shown along the diagonal, along with the
    median estimate and the 90\% credible region (shown with vertical dashed
    lines). \textit{Bottom}: ZTF light curve for ZTF18abgmcmv showing the
    \gztf\ (filled, green circles) and \rztf\ (open, red circles) evolution of
    the SN in the month prior to \tbmax. Observations included in the model
    fitting (i.e., those with $f \le 0.4 f_\mathrm{max}$) are dark and solid,
    while those that are not included are faint and semi-transparent. The
    maximum a posteriori model is shown via a thick solid line, while random
    draws from the posterior are shown with semi-transparent dashed lines. The
    vertical dashed line shows the median 1-D marginalized posterior value of
    \tfl, while the thin, light gray vertical line shows \tfl\ for the maximum
    a posteriori model. The bottom panel shows residuals normalized by their
    uncertainties (pull) relative to the maximum a posteriori model, where the
    factor $\beta_d$ has been included in the calculation of the pull.}
    %
    \label{fig:corner_good}
\end{figure*}

\begin{figure*}
    \centering
    \includegraphics[width=5.2in]{./figures/Fig2.pdf}
    %
    \caption{Same as Figure~\ref{fig:corner_good} for ZTF18abukmty
    (SN\,2018lpz), a typical SN in our sample.}
    %
    \label{fig:corner_median}
\end{figure*}

\begin{figure*}
    \centering
    \includegraphics[width=5.2in]{./figures/Fig3.pdf}
    %
    \caption{Same as Figure~\ref{fig:corner_good} for ZTF18aazabmh
    (SN\,2018crr), a SN that does not significantly constrain the model
    parameters.}
    %
    \label{fig:corner_bad}
\end{figure*}

The bottom panels of Figures~\ref{fig:corner_good}, \ref{fig:corner_median},
and \ref{fig:corner_bad} display the light curves for the corresponding corner
plots shown in the top panels. In addition to the data, we also show multiple
models based on random draws from the posterior, and the residuals normalized
by their uncertainties (pull) of the data relative to the maximum a posteriori
estimate from the MCMC sampling. As illustrated in
Figure~\ref{fig:corner_good}, we can place tight constraints on the model
parameters for light curves with a high SNR. These SNe are typically found at
low redshift, monitored with good sampling and at high photometric precision.
As expected, as the SNR decreases (Figure~\ref{fig:corner_median}) or the
typical interval between observations increases (Figure~\ref{fig:corner_bad}),
it becomes more and more difficult to place meaningful constraints on \tfl\ or
$\alpha$. We visually examine posterior models for each light curve and flag
those that produce unreliable parameter constraints. We use this subset of
sources to identify SNe that should be excluded from the full sample analysis
described in \S\ref{sec:mean_parameters} below (see \ref{sec:qa}). These
flagged sources are noted in Table~\ref{tab:uninformative}.

\section{The Mean Rise Time and Power-Law Index for SNe
Ia}\label{sec:mean_parameters}

Below we examine the results from our model fitting procedure to investigate
several photometric properties of \textit{normal} SNe Ia. We define normal via
the spectroscopic classifications presented in \citet{Yao19}. SNe classified
as SN\,1986G-like, SN\,2002cx-like, Ia-CSM, and super-Chandrasekhar explosions
are excluded from the analysis below. These 7 peculiar events are discussed in
detail in \S\ref{sec:rare}. The remaining 120 normal SNe Ia in our sample
have $-2 \la x_1 \la 2$, where $x_1$ is the \texttt{SALT2} shape parameter,
which is well within the range of SNe that are typically used for cosmography
(e.g., \citealt{Scolnic18a}). Estimates of \trise\ and $\alpha$ for all SNe in
our sample, including the 7 peculiar SNe Ia discussed in \S\ref{sec:rare}, are
presented in Table~\ref{tab:uninformative}.

\subsection{Mean Rise Time of SNe Ia}\label{sec:mean_rise}

From the marginalized 1-D posteriors for \tfl, we can examine the typical rise
time for SNe Ia. The model given in Equation~\ref{eqn:flux_model} constrains
\tfl, yet we ultimately care about the rise time, \trise. \tfl\ is measured
relative to \tbmax, which itself has some measurement uncertainty.\footnote{In
this study \trise\ represents the rise time to $B$-band maximum, as we measure
time relative to \tbmax\ and assume \tfl\ is the same in the $B$, \gztf, and
\rztf\ filters (see \S\ref{sec:model}).} An estimate of \trise\ must therefore
account for the uncertainties on both \tfl\ and \tbmax. \citet{Aldering00}
critically showed that ignoring the uncertainties on the time of maximum could
lead to \trise\ estimates that are incorrect by $\ga 2$\,d.

To measure \trise, we use a Gaussian kernel density estimation (KDE) to
approximate the 1-D marginalized probability density function (PDF) for \tfl.
The width of the kernel is determined via cross validation and the KDE is
implemented with \texttt{scikit-learn} \citep{Pedregosa11}. The PDF is
multiplied by $-1$ and convolved with a Gaussian with the same variance as the
uncertainties on \tbmax\ to determine the final PDF for \trise. The cumulative
density function (CDF) of this PDF is used to determine the median and 90\%
credible region on \trise. We assume there is no significant covariance in the
uncertainties on \tfl\ and \tbmax. These times are estimated using independent
methods and portions of the light curve, which is why we can convolve the
uncertainties in making the final estimation of \trise.

\begin{figure}
    \centering
    \includegraphics[width=1\linewidth]{./figures/rise_time.pdf}
    %
    \caption{Marginalized posterior distribution of the rise time
    $t_\mathrm{rise}$ for normal SNe Ia in our sample. The sample has been
    divided into three groups \edit1{(as described \ref{sec:qa}): thick,
    orange lines show SNe from the reliable-$z_\mathrm{host}$ group (top
    panel), dark blue lines show SNe from the reliable-$z_\mathrm{SN}$ group
    (middle panel), and thin gray lines show SNe from the unreliable group
    (bottom panel).} From the individual PDFs it is clear that there is no
    support for a single mean $t_\mathrm{rise}$ to describe every SN in the
    sample.}
    %
    \label{fig:rise_time}
\end{figure}

In Figure~\ref{fig:rise_time} we show the PDF for \trise\ for the 120 normal
SNe in our sample. We highlight three subsets of the normal SNe in
Figure~\ref{fig:rise_time}: SNe with reliable model parameters
(see~\ref{sec:qa}) and known host-galaxy redshifts (hereafter the
reliable-$z_\mathrm{host}$ group), SNe with reliable model parameters and
unknown host-galaxy redshifts (hereafter the reliable-$z_\mathrm{SN}$ group;
the reliable-$z_\mathrm{host}$ and reliable-$z_\mathrm{SN}$ groups together
form the reliable group), and SNe with large uncertainties in the model
parameters, typically due to sparse sampling around \tfl\ or low photometric
precision (hereafter the unreliable group).

Figure~\ref{fig:rise_time} shows that \trise\ is typically several days
shorter for SNe in the unreliable group relative to SNe in the reliable group.
This provides another indication that the low-quality light curves in our
sample are insufficient for constraining the model parameters.
Figure~\ref{fig:rise_time} also reveals that the rise time among individual
SNe does not tend towards a common mean value. If all SNe Ia could be
described with a single rise time, we could estimate that mean value by
multiplying together each of the PDFs shown in Figure~\ref{fig:rise_time}. The
product of the individual PDFs provides no support for a single rise time to
describe all SNe Ia (i.e., it is effectively equal to 0 everywhere).

As a population, SNe Ia have a mean \trise\,$\approx\,17.4$\,d, where we have
estimated this value by taking a weighted mean of the median value of the
\trise\ PDFs, with weights equal to the square of the inverse of the 68\%
credible region. The mean \trise\ increases to $\approx 18.0$\,d and $18.3$\,d
when considering only the reliable and reliable-$z_\mathrm{host}$ subsamples,
respectively (see Table~\ref{tab:mean_params}). The scatter, estimated via the
sample standard deviation, about these mean values is $\sim$1.8\,d.

\begin{deluxetable*}{lllccccccccccccccccc}
\tabletypesize{\scriptsize}
\tablewidth{0pt}
\tablecaption{90\% Credible Regions for Marginalized Model Parameters (Uninformative Prior)\label{tab:uninformative}}
\tablehead{
\colhead{}
& \colhead{}
& \colhead{}
& \multicolumn{3}{c}{$t_\mathrm{rise}$\,(d)}
& \colhead{}
& \multicolumn{3}{c}{$\alpha_g$}
& \colhead{}
& \multicolumn{3}{c}{$\alpha_r$}
& \colhead{}
& \multicolumn{3}{c}{$\alpha_r - \alpha_g$}
& \colhead{}
& \colhead{} \\
\cline{4-6}
\cline{8-10}
\cline{12-14}
\cline{16-18}
\colhead{ZTF Name}
& \colhead{TNS Name}
& \colhead{$z$\tablenotemark{\footnotesize{a}}}
& \colhead{5}
& \colhead{50}
& \colhead{95}
& \colhead{}
& \colhead{5}
& \colhead{50}
& \colhead{95}
& \colhead{}
& \colhead{5}
& \colhead{50}
& \colhead{95}
& \colhead{}
& \colhead{5}
& \colhead{50}
& \colhead{95}
& \colhead{reliable\tablenotemark{\footnotesize{b}}}
& \colhead{normal\tablenotemark{\footnotesize{c}}}
}
\startdata
ZTF18aailmnv & SN\,2018ebo & 0.080 & 14.23 & 14.91 & 16.29 &  & 0.69 & 1.05 & 1.81 &  & 0.41 & 0.74 & 1.36 &  & -0.86 & -0.34 & 0.09 & n &y \\
ZTF18aansqun & SN\,2018dyp & 0.0597 & 12.45 & 13.69 & 16.33 &  & 1.15 & 3.24 & 5.66 &  & 0.23 & 0.73 & 1.88 &  & -3.06 & -2.81 & -1.46 & n &y \\
ZTF18aaoxryq & SN\,2018ert & 0.0940 & 13.30 & 14.06 & 15.42 &  & 0.31 & 0.64 & 1.10 &  & 0.14 & 0.41 & 0.84 &  & -0.65 & -0.22 & 0.20 & n &y \\
ZTF18aapqwyv & SN\,2018bhc & 0.0560 & 14.02 & 15.07 & 17.05 &  & 1.61 & 2.55 & 4.28 &  & 0.54 & 1.52 & 3.31 &  & -1.98 & -0.97 & 0.03 & n &y \\
ZTF18aapsedq & SN\,2018bgs & 0.0720 & 17.56 & 18.54 & 19.73 &  & 1.62 & 2.05 & 2.62 &  & 1.61 & 3.20 & 5.79 &  & -0.00 & 1.47 & 3.18 & n &y \\
ZTF18aaqcozd & SN\,2018bjc & 0.0732 & 10.85 & 12.13 & 16.53 &  & 0.56 & 2.28 & 4.61 &  & 1.13 & 3.47 & 5.46 &  & 0.63 & 0.81 & 0.99 & n &y \\
ZTF18aaqcqkv & SN\,2018lpc & 0.1174 & 13.16 & 14.79 & 15.99 &  & 0.62 & 2.51 & 5.21 &  & 0.21 & 1.64 & 3.69 &  & -1.74 & 0.80 & 1.82 & n &y \\
ZTF18aaqcqvr & SN\,2018bvg & 0.0716 & 13.69 & 14.32 & 15.62 &  & 0.41 & 0.69 & 1.32 &  & 0.52 & 0.89 & 1.73 &  & 0.00 & 0.25 & 0.52 & n &y \\
ZTF18aaqcugm & SN\,2018bhi & 0.0619 & 13.79 & 15.10 & 17.06 &  & 1.23 & 2.00 & 3.00 &  & 1.01 & 1.65 & 2.49 &  & -0.74 & -0.33 & 0.02 & n &y \\
ZTF18aaqffyp & SN\,2018bhr & 0.070 & 11.76 & 16.21 & 19.98 &  & 0.03 & 0.31 & 1.35 &  & 0.02 & 0.23 & 1.10 &  & -1.15 & -0.06 & 0.83 & n &y \\
ZTF18aaqnrum & SN\,2018bhs & 0.066 & 11.64 & 14.52 & 17.75 &  & 0.14 & 1.64 & 2.97 &  & 0.36 & 2.88 & 4.59 &  & -2.20 & 0.53 & 2.68 & n &y \\
ZTF18aaqqoqs & SN\,2018cbh & 0.082 & 18.31 & 18.85 & 19.67 &  & 1.08 & 1.33 & 1.72 &  & 1.06 & 1.39 & 1.88 &  & -0.17 & 0.06 & 0.33 & y &y \\
ZTF18aarldnh & SN\,2018lpd & 0.1077 & 14.07 & 15.14 & 17.17 &  & 1.20 & 2.22 & 4.21 &  & 0.77 & 1.33 & 2.37 &  & -1.71 & -1.20 & -0.37 & n &y \\
ZTF18aarqnje & SN\,2018bvd & 0.117 & 14.65 & 16.55 & 18.23 &  & 1.27 & 1.97 & 3.24 &  & 0.63 & 1.36 & 2.62 &  & -1.53 & -0.62 & 0.27 & n &y \\
ZTF18aasdted & SN\,2018big & 0.0181 & 18.76 & 18.91 & 19.08 &  & 1.46 & 1.54 & 1.63 &  & 1.30 & 1.39 & 1.50 &  & -0.21 & -0.15 & -0.09 & y &y \\
\enddata
\tablecomments{
This table is available in its entirety in a machine-readable 
form in the online journal. A portion is shown here for guidance 
regarding its form and content. \\}
The table includes the $5^\mathrm{th}$, $50^\mathrm{th}$, 
and $95^\mathrm{th}$ percentiles for the 4 parameters of interest: 
$t_\mathrm{rise}$, $\alpha_g$, $\alpha_r$, $\alpha_r - \alpha_g$. 
90\% credible regions are obtained by subtracting the $5^\mathrm{th}$ percentile 
from the $95^\mathrm{th}$ percentile. Estimates for $t_\mathrm{rise}$ come from 
$t_\mathrm{fl}$ while accounting for the uncertainties on $t_{B,\mathrm{max}}$, 
while estimates for the $\alpha$ parameters have been corrected to a flat prior
(see text for further details).

\tablenotetext{a}{Redshifts are reported to 4 decimal places, 
if the SN host galaxy redshift ($z_\mathrm{host}$) is known. 
Otherwise, the SN redshift ($z_\mathrm{SN}$) is reported to 3 decimal places 
(see \citealt{Yao19} for further details).}
\tablenotetext{b}{Flag for SNe with reliable model parameters (see \ref{sec:qa}). 
y = reliable group, n = unreliable group (see text).}
\tablenotetext{c}{Flag for normal SNe Ia,
y = normal, n = peculiar (the 7 peculiar SNe Ia in our sample are discussed in \ref{sec:rare}; their rise times are measured relative to $T_{g,\mathrm{max}}$).}
\end{deluxetable*}


\subsection{Mean Power-Law Index of the Early Rise}

We use a similar procedure to report the PDF of $\alpha_g$ and $\alpha_r$
under the assumption of a flat prior. The posterior samples for $\alpha$ shown
in Figure~\ref{fig:corner_good}, \ref{fig:corner_median}, and
\ref{fig:corner_bad} include a factor of $10^{-\alpha}$ following the
change of variables from $A$ to $A^\prime$ (see Table~\ref{tab:priors} and
\ref{sec:qa}). To remove this factor, we estimate the 1-D marginalized PDF of
$\alpha$ using a KDE as above. This PDF is then divided by $10^{-\alpha}$, and
then re-normalized to integrate to 1 on the interval from 0 to 10. This final
normalized PDF provides an estimate of $\alpha_g$ and $\alpha_r$ assuming a
$\mathcal{U}(0,10)$ prior.

\begin{figure}
    \centering
    \includegraphics[width=1\linewidth]{./figures/alpha_g.pdf}
    %
    \caption{Marginalized posterior distribution of the rising power-law index
    in the \gztf-band, $\alpha_g$, assuming a flat prior on $\alpha_g$ for
    individual SNe in our sample. The \edit1{panels and color scheme are} the
    same as in Figure~\ref{fig:rise_time}. While the density of the PDFs tends
    towards 2, there is no support for a single mean power-law index to
    describe all SNe Ia.}
    %
    \label{fig:alpha_rise}
\end{figure}

The PDFs for $\alpha_g$ for normal SNe Ia are shown in
Figure~\ref{fig:alpha_rise}. The tightest constraints on $\alpha_g$ come from
the reliable-$z_\mathrm{host}$ group, which are clustered around $\alpha_g
\approx 2$. There are, however, individual reliable-$z_\mathrm{host}$ SNe that
provide support for $\alpha_g$ as low as $\sim$0.7 and as high as $\sim$3.5, meaning $\alpha_g$ can take on a wide range of values.

The weighted sample mean is $\alpha_g \approx 1.9$ for normal SNe Ia in the
ZTF sample. This value increases to $\sim$2.1 when reducing the sample to the
reliable group or the reliable-$z_\mathrm{host}$ group. The population scatter
is $\sim$0.6 (see Table~\ref{tab:mean_params}). For $\alpha_r$ the weighted
sample mean is $\sim$1.7, $\sim$1.9, and $\sim$2.0 for the full sample, the
reliable group, and reliable-$z_\mathrm{host}$ group, respectively. The
typical scatter in $\alpha_r$ is 0.5 (see Table~\ref{tab:mean_params}). As
noted in \S\ref{sec:alpha_correlation}, there is a tight correlation
between $\alpha_g$ and $\alpha_r$, and thus we do not show the individual PDFs
for $\alpha_r$.

In both the \gztf\ and \rztf\ filters the mean rising power-law index for the
initial evolution of the SN is close to 2, as might be expected in
the expanding fireball model. While the mean value of $\alpha$
is $\sim$2, it is noteworthy that the PDF for several SNe in the
reliable-$z_\mathrm{host}$ sample provide no support for $\alpha = 2$. If we
multiply the individual PDFs of $\alpha_g$ or $\alpha_r$ together we find
there is no support for a single mean value of $\alpha$ capable of explaining
every SN in our sample. This suggests that models using a fixed value of
$\alpha$ are insufficient to explain the general population of normal SNe Ia
(though see also \S\ref{sec:strong_priors}).

\begin{deluxetable*}{lrcccccccccrcc}
\tabletypesize{\scriptsize}
\tablewidth{0pt}
\tablecaption{Population Mean and Scatter For $t_\mathrm{rise}$ and $\alpha$\label{tab:mean_params}}
\tablehead{
\colhead{}
& \colhead{}
& \multicolumn{8}{c}{Uninformative prior}
& \colhead{}
& \multicolumn{3}{c}{$\alpha = 2$ prior} \\
\cline{2-10}
\cline{12-14}
\colhead{Subset}
& \colhead{N}
& \colhead{$t_\mathrm{rise}$\,(d)}
& \colhead{$\sigma_{t_\mathrm{rise}}$\,(d)}
& \colhead{$\alpha_g$}
& \colhead{$\sigma_{\alpha_g}$}
& \colhead{$\alpha_r$}
& \colhead{$\sigma_{\alpha_r}$}
& \colhead{$\alpha_r - \alpha_g$}
& \colhead{$\sigma_{\alpha_r - \alpha_g}$}
& \colhead{}
& \colhead{N}
& \colhead{$t_\mathrm{rise}$\,(d)}
& \colhead{$\sigma_{t_\mathrm{rise}}$\,(d)}
}
\startdata
normal &                    120 & $ 17.41\pm0.04 $ & 1.81 & $ 1.89\pm0.02 $ & 0.75 & $ 1.73\pm0.02 $ & 0.80 & $ -0.18\pm0.01 $ & 0.73 && 120 & $ 21.03\pm0.02 $ & 1.75 \\
reliable &                   47 & $ 18.05\pm0.05 $ & 1.60 & $ 2.05\pm0.02 $ & 0.53 & $ 1.89\pm0.02 $ & 0.50 & $ -0.17\pm0.01 $ & 0.23 && 115 & $ 21.03\pm0.02 $ & 1.75 \\
reliable-$z_\mathrm{host}$ & 25 & $ 18.29\pm0.05 $ & 1.83 & $ 2.12\pm0.02 $ & 0.59 & $ 1.99\pm0.03 $ & 0.54 & $ -0.18\pm0.02 $ & 0.17 &&  58 & $ 21.01\pm0.02 $ & 1.96 \\
\hline
\multicolumn{14}{c}{Volume-limited ($z < 0.06$) subset} \\
\hline
normal &                     28 & $ 18.18\pm0.05 $ & 2.23 & $ 2.05\pm0.03 $ & 0.76 & $ 1.95\pm0.03 $ & 0.67 & $ -0.21\pm0.01 $ & 0.86 && 28 & $ 21.17\pm0.02 $ & 1.46 \\
reliable &                   16 & $ 18.48\pm0.05 $ & 1.73 & $ 2.13\pm0.03 $ & 0.54 & $ 2.01\pm0.03 $ & 0.52 & $ -0.18\pm0.02 $ & 0.08 && 27 & $ 21.17\pm0.02 $ & 1.43 \\
reliable-$z_\mathrm{host}$ & 15 & $ 18.52\pm0.05 $ & 1.68 & $ 2.14\pm0.03 $ & 0.51 & $ 2.02\pm0.03 $ & 0.50 & $ -0.18\pm0.02 $ & 0.09 && 24 & $ 21.16\pm0.02 $ & 1.47 \\
DIC preferred &  & $ \ldots $ & $ \ldots $ & $ \ldots $ & $ \ldots $ & $ \ldots $ & $ \ldots $ & $ \ldots $ & $ \ldots $              && 28 & $ 19.45\pm0.03 $ & 1.37 \\
DIC--uninformative &          9 & $ 18.83\pm0.03 $ & 1.33 & $ 2.14\pm0.03 $ & 0.55 & $ 2.02\pm0.03 $ & 0.52 & $ -0.19\pm0.02 $ & 0.08 && $ \ldots $ & $ \ldots $ & $ \ldots $ \\
\enddata
\tablecomments{
Table includes the weighted mean (see text), plus standard uncertainty in the weighted mean, 
as well as the scatter (the sample standard deviation), for the 4 parameters of interest, 
$t_\mathrm{rise}$, $\alpha_g$, $\alpha_r$, $\alpha_r - \alpha_g$, for the uninformative 
and $\alpha = 2$ priors. $N$ is the number of SNe in each subset of the data, which are defined as follows 
(see text for more detailed definitions):
normal -- normal SNe Ia, 
reliable -- SNe with reliable model parameters, 
reliable-$z_\mathrm{host}$ -- reliable SNe with known host galaxy redshifts, 
DIC preferred -- results from the $\alpha = 2$ prior, 
\textit{unless the DIC prefers the uninformative prior} (see \S\ref{sec:dic}; only applies to $t_\mathrm{rise}$),
DIC--uninformative -- only SNe where the DIC prefers the uninformative prior 
(see \S\ref{sec:dic}; excludes the $\alpha = 2$ prior by construction).
The volume limited subset includes only SNe with $z < 0.06$ (see \S\ref{sec:volume_limited}).
Note that the  definition of reliable differs for the uninformative and $\alpha = 2$ priors, 
see \ref{sec:qa} and \S\ref{sec:strong_priors}, respectively.
}

\end{deluxetable*}


\subsection{Mean Color Evolution}\label{sec:colors}

Here we examine the mean initial color evolution of SNe Ia, \textit{under the
assumption that the early emission from SNe Ia can correctly be described by
the power-law model adopted in \S\ref{sec:model}}. This analysis does not
address the initial colors of SNe Ia; for a more detailed analysis of the
initial colors and color evolution of SNe Ia see Paper III in this series
\citep{Bulla20}.

Unlike \trise\ and $\alpha$, we do find evidence for a single mean value of
the early color evolution of SNe Ia, as traced by $\alpha_r - \alpha_g$. If
the early evolution in the \gztf\ and \rztf\ filters is a power-law in time,
then the \gztf\ $-$ \rztf\ color, in mag, will be proportional to $(\alpha_r -
\alpha_g) \log_{10} (t - t_\mathrm{fl})$.

\begin{figure}
    \centering
    \includegraphics[width=1\linewidth]{./figures/delta.pdf}
    %
    \caption{Marginalized posterior distribution of the early SN Ia color
    evolution, $\alpha_r - \alpha_g$, assuming flat priors on $\alpha_g$ and
    $\alpha_r$. The \edit1{panels and color scheme are} the same as in
    Figure~\ref{fig:rise_time}. The thick, solid black line shows an estimate
    of the mean value of $\alpha_r - \alpha_g$, which is estimated by
    multiplying together the likelihoods for SNe \textit{in the
    reliable-$z_\mathrm{host}$ group}, \edit1{which is why this mean is only
    shown in the top panel}. There is support for a single mean value of
    $\alpha_r - \alpha_g \approx -0.18$ (see Table~\ref{tab:mean_params}).}
    %
    \label{fig:delta}
\end{figure}

To estimate $\alpha_r - \alpha_g$ we use a similar procedure as above,
however, we need to estimate the marginalized joint posterior on $\alpha_g$
and $\alpha_r$, $\pi(\alpha_g,\alpha_r \mid t_\mathrm{fl}, A^\prime_b,
\beta_d)$, in order to correct the posterior estimates for the priors on
$\alpha$. We estimate the 2D joint posterior via a Gaussian KDE, correct this
distribution for the priors on $\alpha_g$ and $\alpha_r$, and then obtain
random draws from this distribution to estimate the 1-D marginalized
likelihood estimates on $\alpha_r - \alpha_g$. The PDFs for $\alpha_r -
\alpha_g$ for individual SNe are shown in Figure~\ref{fig:delta}.

Unlike the estimates for \trise\ and $\alpha$ alone, $\alpha_r - \alpha_g$ is
clearly clustered around $\sim$${-0.2}$ for the reliable group. Multiplying
these likelihoods together produces support for a single mean value of
$\alpha_r - \alpha_g = -0.169 \pm 0.015$, where the uncertainties on that
estimate represent the 90\% credible region. The mean PDF is shown as the
thick, solid black line in Figure~\ref{fig:delta}. A mean value of $\alpha_r -
\alpha_g$ suggests that a typical, normal SN Ia becomes bluer in the days
after explosion. Such an evolution makes sense for an optically thick,
radioactively heated, expanding ejecta (e.g., \citealt{Piro16,Magee20}). There
are, however, clear examples of individual SNe that do not exhibit this
behavior (e.g., SN\,2017cbv and iPTF\,16abc;
\citealt{Hosseinzadeh17,Miller18}), meaning this mean behavior is not
prescriptive for every SN Ia. These results exclude SNe from the unreliable
group, and their inclusion would remove any support for a single mean value of
$\alpha_r - \alpha_g$. This is largely due to a small handful of events that
feature extreme values of $\alpha_r - \alpha_g$ because there are gaps in the
observational coverage of one of the two filters (see the upper right panel of
Figure~\ref{fig:model_parameters}).

\section{Population correlations}

In addition to looking at the typical values of \trise\ and $\alpha$ for SNe
Ia, we also examine the correlations between these parameters, as well as how
they evolve with redshift, $z$. These correlations may reveal details about
the explosion physics of SNe Ia (for example, if strong mixing in the SN
ejecta affects the early evolution, as found in \citealt{Piro16},
\citealt{Magee18}, and \citealt{Magee20}, then any correlations with $\alpha$
may be related to ejecta mixing). If the model parameters are correlated with
redshift, that could be evidence for either cosmic evolution of SNe Ia
progenitors or inadequacies in the model.

The correlation between $t_\mathrm{rise}$, $\alpha_g$, $\alpha_r$, and $z$ is
shown in Figure~\ref{fig:model_parameters}. We do not show the correlation
between $\alpha_r$ and $z$ or between $\alpha_g$ and \trise, as this
information is effectively redundant given the tight correlation between
$\alpha_g$ and $\alpha_r$ (top right panel of
Figure~\ref{fig:model_parameters}).

\begin{figure*}
    \centering
    \includegraphics[width=6in]{./figures/param_correlations.pdf}
    %
    \caption{Correlation between redshift, $z$, SN rise time,
    $t_\mathrm{rise}$, and the power-law index in the \gztf\ and \rztf\
    filters, $\alpha_g$ and $\alpha_r$, respectively. We do not show
    $\alpha_r$ vs.~$z$ or $t_\mathrm{rise}$ vs.~$\alpha_g$, as these would
    largely be redundant given the very strong correlation between $\alpha_g$
    and $\alpha_r$ (upper right panel). The sample has been divided into three
    groups: small, light gray circles show SNe from the unreliable group (see
    \ref{sec:qa}), dark blue circles show the reliable-$z_\mathrm{SN}$ group,
    and large orange circles show the reliable-$z_\mathrm{host}$ group. The
    plots show that redshift is correlated with both $t_\mathrm{rise}$ and
    $\alpha_g$, which would only be expected if SNe Ia undergo significant
    evolution from $z \approx 0$ to $0.1$. We show this to be the result
    of a systematic selection effect (see text for further details). }
    %
    \label{fig:model_parameters}
\end{figure*}

\subsection{Correlation Between $\alpha_g$ and $\alpha_r$}\label{sec:alpha_correlation}

The most striking feature in Figure~\ref{fig:model_parameters} is the tight
correlation between $\alpha_g$ and $\alpha_r$. This result is
reasonable because the SN SED is approximately a
black body, and the \gztf\ and \rztf\ filters are relatively line free
(compared to the UV) and sample adjacent portions of the Rayleigh-Jeans tail.
Thus the evolution should be nearly identical in the two filters. SNe with
reliable model parameters follow a tight locus around $\alpha_r - \alpha_g
\approx -0.2$, with the only major outliers from this relation being SNe in
the unreliable group.

The Spearman rank-ordered correlation coefficient for $\alpha_g$ and
$\alpha_r$ is highly significant for the entire population ($\rho > 0.5$).
Restricting the sample to SNe with reliable model parameters increases the
significance of the correlation dramatically ($\rho > 0.9$). Thus, knowledge
of the power-law index in either filter provides a strong predictor for the
power-law index in the other filter.

\subsection{Correlations with Redshift -- Systematics, Not Cosmic Evolution}\label{sec:redshift_correlations}

While less prominent, Figure~\ref{fig:model_parameters} additionally shows
that both \trise\ and $\alpha$ are correlated with redshift. This result is
somewhat surprising: naively, it suggests some form of cosmic evolution in SNe
Ia, with SNe at $z \approx 0.08$ having rise times that are several days
shorter than SNe at $z \approx 0.02$. The small range of redshifts in our
sample, and several previous studies (e.g.,
\citealt{Aldering00,Conley06,Jones19}), render this naive explanation in
doubt. Instead, these correlations are the result of building a sample from a
flux-limited survey.

Given that ZTF cannot detect SNe when their observed brightness is
$g_\mathrm{ZTF} \ga 21.5$\,mag \citep{Masci19,Bellm19}, SNe at progressively
higher redshifts are discovered at a later phase in their evolution. The large
degeneracies in the model presented in Equation~\ref{eqn:flux_model}, namely
between \tfl, $A$, and $\alpha$, allow for a great deal of flexibility when
fitting the data. For SNe discovered at later phases, it is possible to adjust
\tfl\ while decreasing $A$ and $\alpha$, such that \tfl\ occurs around the
epoch of first detection (resulting in a shorter rise time).

We illustrate this effect in Figure~\ref{fig:high_z_systematic}, which shows
that the inferred rise time for identical SNe decreases as those SNe are
observed at successively higher redshifts. We use the 4 normal SNe with $z \le
0.03$ and simulate their appearance at higher redshift by making the
(over-simplified) assumption that all detections are in the sky-background
dominated regime. Thus, in any given epoch the $\mathrm{SNR} \propto
d_L^{-2}$, where $d_L$ is the SN luminosity distance. To simulate the SN at
some new redshift, $z_\mathrm{sim}$, we multiply the uncertainties by
$(d_{L,\mathrm{sim}}/d_{L,\mathrm{obs}})^2$, where $d_{L,\mathrm{sim}}$ is the
luminosity distance at $z_\mathrm{sim}$, and $d_{L,\mathrm{obs}}$ is the
observed luminosity distance to the SN.\footnote{Following \citet{Yao19}, we
adopt a flat $\Lambda$CDM cosmology with $H_0 =
73.24$\,km\,s$^{-1}$\,Mpc$^{-1}$ \citep{Riess16} and $\Omega_m = 0.275$
\citep{Amanullah10} to calculate $d_L$ for the SNe.} Using these increased
uncertainties, we randomly resample the observed flux values from a normal
distribution with mean equal to the original flux and variance equal to the
square of the distance-scaled uncertainty. After correcting the observation
times to the simulated rest frame, we fit the noisier simulated data with the
procedure from \S\ref{sec:model}. We simulate the appearance of these SNe at
redshifts $z = 0.05$, 0.075, 0.1, and 0.15. Only the models that converge are
shown in Figure~\ref{fig:high_z_systematic}.

\begin{figure}
    \centering
    \includegraphics[width=1\linewidth]{./figures/high_z_systematic.pdf}
    %
    \caption{Same as the bottom left panel of
    Figure~\ref{fig:model_parameters}, though all SNe are shown in gray. The
    large green circle, magenta diamond, orange triangle, and purple $X$ show
    how marginalized posterior estimates of \trise\ change as the four lowest
    redshift SNe are observed at $z = 0.05$, 0.075, 0.1, and 0.15 (see text
    for further details). For clarity, slight offsets in $z$ have been applied
    to the symbols, as the error bars would otherwise fully overlap. \trise\
    clearly decreases with increasing redshift, showing that the observed
    correlation between these parameters is a consequence of flux-limited SN
    surveys.}
    %
    \label{fig:high_z_systematic}
\end{figure}

The results shown in Figure~\ref{fig:high_z_systematic} are clear: SNe
discovered at higher redshifts have systematically smaller estimates for
\trise. This result is simple to understand as higher redshift SNe will not be
detected until later in their evolution. A stronger prior on any of the model
parameters would help to combat this effect (see \S\ref{sec:strong_priors}),
though as previously discussed we avoid strong priors due to the wide range of
$\alpha$ and \tfl\ that has been reported in the literature. 

This effect also explains the correlation seen in the bottom right panel of
Figure~\ref{fig:model_parameters}. SNe detected later in their evolution will
be evolving less rapidly as the rate of change in brightness continually
decreases until the time of maximum light. Hence, a later detection provides a
lower value of $\alpha$. Indeed, a re-creation of
Figure~\ref{fig:high_z_systematic} showing $\alpha_g$ instead of \trise\ shows
$\alpha_g$ decreasing with increasing redshift. Thus, the observed
correlations with redshift seen in Figure~\ref{fig:model_parameters} can be
entirely understood as the result of ZTF being a flux-limited survey.

The implications of this result have consequences well beyond the ZTF sample
discussed here. Essentially all SN surveys are flux-limited, meaning the
systematics associated with redshift will affect any efforts to determine
\trise\ or $\alpha$ in those data as well. The inclusion of higher-redshift
SNe in the sample will, on average, bias estimates of \trise\ and $\alpha$ to
lower values. Even more concerning is the possibility that this trend may
continue to very low redshifts ($z \ll 0.01$). The paucity of SNe in this
redshift range, due to the relatively small volume probed, make it difficult
to test for such an effect. Due to the systematic identified here, it may be
the case that the rise time, and by extension also $\alpha$, are
underestimated for every SN in the literature. Detailed simulations with
realistic SN light curves are needed to test this possibility.

\subsection{Correlation Between \trise\ and $\alpha$}

The lower right panel of Figure~\ref{fig:model_parameters} shows that \trise\
and $\alpha_r$ are corerlated (and by extension \trise\ and $\alpha_g$ are
also correlated). The Spearman correlation coefficient for \trise\ and
$\alpha_r$ is significant ($\rho > 0.5$) for both the entire population of SNe
in this study and the reliable subset as well. Similar values are found for
\trise\ and $\alpha_g$.

The origins of such a correlation may be a consequence of the $^{56}$Ni
distribution in the SN ejecta. In \citet{Magee20} a suite of models is
developed to explore the effects of $^{56}$Ni mixing on the resulting emission
from SNe Ia. The full family of models in \citet{Magee20}, which was designed
to cover a wide range of parameter space and not the physical space occupied
by observed SNe Ia, does not show significant correlation between \trise\ and
$\alpha_r$. When including only models that do a good job of reproducing
observations (see Section 5 in \citealt{Magee20}), there is a strong
correlation between \trise\ and $\alpha_r$ (Spearman $\rho > 0.9$) that
roughly matches the slope in Figure~\ref{fig:model_parameters} (M.~Magee,
private communication). It is therefore possible that the observed correlation
between \trise\ and $\alpha_r$ is the result of thermonuclear explosions
producing different $^{56}$Ni distributions.

\subsection{Correlations with Light Curve Shape}

A defining characteristic of SNe Ia is that they can be described by a
relatively simple luminosity-shape relation \citep{Phillips93}, which enables
them to be used as standardizable candles. We examine the correlation between
light curve shape, in this case the \texttt{SALT2} $x_1$ parameter, and the SN
rise time and $\alpha$ in Figure~\ref{fig:shape_correlations}. There is a clear
correlation between shape and \trise, which has been hinted at in other smaller
samples (e.g., \citealt{Riess99a,Firth15,Zheng17a}). The Spearman coefficient
for $x_1$ and \trise\ is significant for the entire population ($\rho > 0.4$),
and increases when considering the reliable-$z_\mathrm{host}$ group ($\rho >
0.6$).

An observed correlation between $x_1$ and \trise\ should be expected as the
$x_1$ shape parameter accounts for the width of both the SN rise and decline,
and therefore, by definition, should be correlated with the rise time. The
middle and right panels of Figure~\ref{fig:shape_correlations} divide the
reliable-$z_\mathrm{host}$ group into low ($z < 0.06$) and high ($z \ge 0.06$)
redshift bins. From these panels it is clear that some of the scatter in the
$x_1$--\trise\ plane is the result of the redshift bias discussed in
\S\ref{sec:redshift_correlations}, as higher redshift SNe have shorter rise
times at fixed $x_1$. A correction for this redshift effect would reduce the
overall scatter seen in the lower panels of
Figure~\ref{fig:shape_correlations} (see \S\ref{sec:strong_priors}).

\begin{figure*}
    \centering
    \includegraphics[width=6in]{./figures/shape_correlations.pdf}
    %
    \caption{Correlation between the \texttt{SALT2} $x_1$ shape parameter and
    $\alpha_g$ (top row) and \trise\ (bottom row). Symbols are the same as in
    Figure~\ref{fig:model_parameters}. For clarity, the uncertainties on $x_1$
    are not shown. \trise\ shows a strong correlation with $x_1$, while there is
    no correlation between $\alpha_g$ and $x_1$. The middle and right panels
    highlight reliable-$z_\mathrm{host}$ SNe at low, $z < 0.06$, and high, $z
    \ge 0.06$, redshift, respectively. Dividing the sample into different
    redshift bins shows that some of the observed scatter between $x_1$ and the
    model parameters is due to redshift and not intrinsic scatter. }
    %
    \label{fig:shape_correlations}
\end{figure*}

\begin{figure*}
    \centering
    \includegraphics[width=6in]{./figures/dm15_rise.pdf}
    %
    \caption{Correlation between $\Delta m_{15}(g_\mathrm{ZTF})$ and \trise.
    Symbols are the same as in Figure~\ref{fig:model_parameters}. There is a
    clear correlation between the rise and decline times of SNe Ia. Some of the
    scatter in the relationship between these parameters can be explained as a
    result of the redshift effect discussed in
    \S\ref{sec:redshift_correlations}.}
    %
    \label{fig:dm15}
\end{figure*}

We compare separate measurements of the rise and decline of SNe Ia in
Figure~\ref{fig:dm15}, which shows the correlation between \trise\ and $\Delta
m_{15}(g_\mathrm{ZTF})$, the observed decline in magnitudes of the \gztf\
light curve between the time of \gztf\ maximum light and 15\,d later. $\Delta
m_{15}(g_\mathrm{ZTF})$ is measured via low-order polynomial fits to the
observed \gztf\ photometry, i.e., $K$-corrections have not been applied, from
a SN rest-frame phase $= -5$ to $+20$\,d. For 15 of the 127 SNe in our sample
this measurement is not possible due to an insufficient number of observations
in the defined window. We additionally exclude 6 SNe from
Figure~\ref{fig:dm15} where the relative uncertainty on $\Delta
m_{15}(g_\mathrm{ZTF})$ is greater than 25\%. Figure~\ref{fig:dm15} shows
that, on average, slowly declining SNe with smaller values of $\Delta
m_{15}(g_\mathrm{ZTF})$ have longer rise times (Spearman $\rho \approx -0.35$
for the full sample, and $\sim$$-0.6$ for the reliable-$z_\mathrm{host}$
group). However, there is not a significant correlation between these
parameters for the lowest-redshift ($z < 0.06$) reliable-$z_\mathrm{host}$ SNe
(middle panel of Figure~\ref{fig:dm15}).

The observed correlation between the rise time and the SN decline rate stands
in contrast to what was found in \citet{Hayden10}, where the slowest declining
SNe are among the fastest risers, and a few fast-declining SNe have long rise
times. Interrogation of the low-$z$ SNe in our sample reveals a
non-significant correlation, however, and individual SNe in our sample conform
with the behavior described in \citet{Hayden10}. For example, ZTF18abkhcrj
(SN\,2018emi) has $\Delta m_{15}(g_\mathrm{ZTF}) \approx 0.7$, making it a
slow-declining SN, yet the rise time is a relatively short $\sim$16\,d. In
this particular case the short rise time may be the result of larger
photometric uncertainties than is typical for ZTF SNe. ZTF18abkhcrj
(SN\,2018emi) was discovered on top of the nucleus of its host galaxy during
full moon. These two affects would contribute additional noise, delaying the
phase at which the SN is discovered, and, similar to the affects discussed in
\S\ref{sec:redshift_correlations}, yield a shorter estimate for \trise. We
conclude that additional examples are needed to better understand the
relationship between rise and decline of SNe Ia.

There is no strong correlation between $\alpha$ and $x_1$
(Figure~\ref{fig:shape_correlations}). The Spearman correlation for these two
parameters is $\rho \approx -0.2$ whether looking at $\alpha_g$ or $\alpha_r$,
or whether considering the full sample, the reliable group, or the
reliable-$z_\mathrm{host}$ group. Subdividing the reliable-$z_\mathrm{host}$
group by redshift shows the same trend that was identified in
Figure~\ref{fig:model_parameters}: higher redshift SNe have smaller values of
$\alpha$ on average for the reliable-$z_\mathrm{host}$ group.

\section{Strong Priors}\label{sec:strong_priors}

\subsection{Fixing $\alpha = 2$}

In our previous effort to model the early evolution of SNe Ia we adopted a
flexible model (hereafter the ``uninformative prior'') allowing $\alpha$ and
\tfl\ to simultaneously vary, despite theoretical \citep{Arnett82,Riess99a} and
observational \citep{Conley06,Hayden10,Ganeshalingam11} evidence that $\alpha$
is consistent with $2$. Here we alter the model by fixing $\alpha_g = \alpha_r
= 2$ (hereafter the ``$\alpha = 2$ prior''), and explore how this decision
changes the results described in the previous sections. This decision is
equivalent to placing an infinitely strong prior on the value of
$\alpha$.\footnote{Strictly enforcing $\alpha_g = \alpha_r$ imposes
non-physical structure on the models, as this condition effectively implies
that there is no change in the $g - r$ color during the initial rise of the
SN. This is clearly observed not to be the case in many SNe
(\S\ref{sec:colors}; see also Paper III in this series, \citealt{Bulla20}).}

The distribution of rise time PDFs using the $\alpha = 2$ prior is shown in
Figure~\ref{fig:tsquared_rise}, and reported in Table~\ref{tab:alpha2_rise}.
Adopting this strict prior significantly reduces the flexibility of the model.
One consequence of this choice is that visual inspection of the posterior
predictive flux values reveals that there are far fewer SNe with unreliable
model parameters. When using the $\alpha = 2$ prior, we only flag SNe with an
extrapolated flux using the maximum a posteriori model parameters $< 0.9
f_\mathrm{max}$ \edit1{at $T_{g,\mathrm{max}}$ as} having unreliable model
parameters. From this criterion only 5 SNe are identified as having unreliable
model parameters.

\begin{deluxetable}{llcccc}
\tabletypesize{\scriptsize}
\tablewidth{0pt}
\tablecaption{Ninety Percent Credible Region for $t_\mathrm{rise}$ ($\alpha = 2$ Prior)\label{tab:alpha2_rise}}
\tablehead{
\colhead{}
& \colhead{}
& \multicolumn{3}{c}{$t_\mathrm{rise}$\,(d)}
& \colhead{} \\
\cline{3-5}
\colhead{ZTF Name}
& \colhead{TNS Name}
& \colhead{5}
& \colhead{50}
& \colhead{95}
& \colhead{Reliable\tablenotemark{\scriptsize{a}}}
}
\startdata
ZTF18aailmnv & SN\,2018ebo & 21.01 & 22.22 & 23.65 & y \\
ZTF18aansqun & SN\,2018dyp & 17.82 & 19.22 & 20.88 & y \\
ZTF18aaoxryq & SN\,2018ert & 21.02 & 23.21 & 26.34 & y \\
ZTF18aapqwyv & SN\,2018bhc & 17.83 & 18.95 & 20.53 & y \\
ZTF18aapsedq & SN\,2018bgs & 21.71 & 22.38 & 23.03 & y \\
ZTF18aaqcozd & SN\,2018bjc & 18.12 & 19.59 & 21.52 & y \\
ZTF18aaqcqkv & SN\,2018lpc & 17.62 & 19.79 & 22.13 & n \\
ZTF18aaqcqvr & SN\,2018bvg & 21.18 & 21.98 & 22.91 & y \\
ZTF18aaqcugm & SN\,2018bhi & 19.06 & 19.52 & 20.04 & y \\
ZTF18aaqffyp & SN\,2018bhr & 17.91 & 23.36 & 26.79 & n \\
ZTF18aaqnrum & SN\,2018bhs & 16.47 & 22.55 & 25.51 & y \\
ZTF18aaqqoqs & SN\,2018cbh & 24.15 & 24.62 & 25.12 & y \\
ZTF18aarldnh & SN\,2018lpd & 19.32 & 20.57 & 22.04 & y \\
ZTF18aarqnje & SN\,2018bvd & 20.40 & 21.55 & 22.83 & y \\
ZTF18aasdted & SN\,2018big & 23.42 & 23.53 & 23.65 & y \\
\enddata
\tablecomments{
The table includes the $5^\mathrm{th}$, $50^\mathrm{th}$, 
and $95^\mathrm{th}$ percentiles for  
$t_\mathrm{rise}$ after adopting the $\alpha = 2$ prior (see text for further details).}

\tablenotetext{a}{Flag for SNe with reliable model parameters.
Note that the $\alpha = 2$ prior definition of reliable differs from that in 
\ref{sec:qa} (see text).}
(This table is available in its entirety in a machine-readable 
form in the online journal.)
\end{deluxetable}


\begin{figure}
    \centering
    \includegraphics[width=1\linewidth]{./figures/tsquared_rise_time.pdf}
    %
    \caption{Same as Figure~\ref{fig:rise_time}, showing the resulting PDFs
    for the $\alpha = 2$ prior. Note that the definition for reliable model
    parameters with the $\alpha = 2$ prior is different from that described in
    \ref{sec:qa} (see text). As was the case when $\alpha$ is allowed to vary,
    there is no support for a single mean $t_\mathrm{rise}$ to describe every
    SN in the sample. The $\alpha = 2$ prior results in rise times that are
    $\sim$3\,d longer on average.}
    %
    \label{fig:tsquared_rise}
\end{figure}

\edit1{Figure~\ref{fig:trise_prior} shows how the inferred rise time changes
when adopting the $\alpha = 2$ prior instead of the uninformative prior for
every SN in our sample. The $\alpha = 2$ prior leads to larger rise times in
each individual SN, with an average increase of $\sim$3\,d in \trise.}
Adopting the $\alpha = 2$ prior also leads to far more SNe with narrow PDFs
for \trise\ \edit1{(compare Figures~\ref{fig:tsquared_rise}
and~\ref{fig:rise_time}).}

Multiplying the individual likelihoods for \trise\ does not provide support
for a single mean rise time. Following the same approach described in
\S\ref{sec:mean_rise}, we find a population mean \trise\;$\approx 21.0$\,d,
with a corresponding population scatter of $\sim$1.8\,d for the $\alpha = 2$
prior (see Table~\ref{tab:mean_params}). Another consequence of adopting the
$\alpha = 2$ prior is that a small handful ($\sim$5--6) of SNe have rise times
consistent with 26\,d, which is considerably longer than the rise times
inferred in any previous study of normal SNe Ia.

\begin{figure}
    \centering
    \includegraphics[width=1\linewidth]{./figures/trise_vs_prior.pdf}
    %
    \caption{\edit1{Comparison of the rise time inferred when adopting the
    uninformative prior and the $\alpha = 2$ prior for the 120 normal SNe in
    our sample. Adoption of the $\alpha = 2$ prior leads to larger values of
    \trise\ for every SN, with an average increase of $\sim$3\,d. The
    individual lines are colored orange, dark blue, and gray for the
    uninformative prior reliable-$z_\mathrm{host}$, reliable-$z_\mathrm{SN}$,
    and unreliable groups, respectively.}}
    %
    \label{fig:trise_prior}
\end{figure}

Figure~\ref{fig:tsquared_z_evolution} shows \trise\ as a function of redshift
(left) and $x_1$ (right) when adopting the $\alpha = 2$ prior. The previously
observed correlation between \trise\ and redshift disappears when assuming
$\alpha = 2$. The Spearman rank-order correlation coefficient for these two
parameters is $\rho \la 0.2$ for the full sample, and the reliable and
reliable-$z_\mathrm{host}$ subsamples. The model is, in effect, no longer
flexible enough to systematically adjust \tfl\ to be approximately equal to
the epoch of first detection. The removal of this particular bias provides a
\edit1{potential} benefit of fixing $\alpha = 2$.

\begin{figure*}
    \centering
    \includegraphics[width=6in]{./figures/trise_z_tsquared.pdf}
    %
    \caption{Correlation between \trise\ and redshift (left) and $x_1$ (right)
    when adopting the $\alpha = 2$ prior for the early emission from SNe Ia.
    The use of this strict prior removes the previously observed bias that
    resulted in shorter rise times being inferred at higher redshifts. A
    consequence of the removal of this bias is a reduction in the observed
    scatter between \trise\ and $x_1$.}
    %
    \label{fig:tsquared_z_evolution}
\end{figure*}


Adopting the $\alpha = 2$ prior yields a significantly smaller scatter in the
correlation between $x_1$ and \trise, as shown in
Figures~\ref{fig:shape_correlations} and~\ref{fig:tsquared_z_evolution}. The
reduction in this scatter intuitively makes sense, given that it was due, at
least in part, to the redshift bias in measuring \trise\
(\S\ref{sec:redshift_correlations}). Reducing, or possibly fully removing,
that bias by adopting the $\alpha = 2$ prior allows a direct estimate of the
rise time from $x_1$ with a typical scatter $< 1$\,d. If relatively high
precision measurements of \trise\ can be directly inferred from $x_1$, it
would dramatically increase the sample of SNe Ia with measured rise times, as
extremely early observations ($t < -10$\,d) would no longer be required (see
\S\ref{sec:x1_rise}).

\subsection{Model Selection}\label{sec:dic}

In adopting two very different priors that, in turn, produce significantly
different posteriors, we are naturally left with the question: which model is
better? To some extent, the answer to this question rests with every individual
as the prior quantifies one's a priori belief about the model parameters. We
posit that it is extremely unlikely that thermonuclear explosions all
identically produce $\alpha = 2$ across a multitude of filters. Thus, adopting
$\alpha = 2$ is very likely an overconfident position that produces slightly
biased inference as a result.

Alternatively, we can address the question of which model is best via the use
of model selection techniques based on information criteria. In our initial
fit of the SN light curves we effectively included additional parameters in
the model by allowing $\alpha$ to vary. Thus, for individual SNe, we can
compare the trade-off between increasing the model complexity relative to the
overall improvement in the model fit to the data, in order to determine which
model is superior. Following \citet{Spiegelhalter02}, we define the deviance
$D$ as
%
$$D(\theta) = -2 \ln (p(x\mid \theta)) + C,$$
%
where $\theta$ are the model parameters, $x$ are the observations, $p(x\mid
\theta)$ is the likelihood, and $C$ is a constant that will drop out following
model comparison. From here, the effective number\footnote{\edit1{Many model
selection techniques based on information criteria, such as the Akaike
information criterion \citep[AIC;][]{Akaike73} or ``Bayesian'' information
criterion \citep[BIC;][]{Schwarz78}, include a point estimate of the maximized
likelihood and then subtract a penalty term related to the total number of
parameters in the model. These methods aim to balance the goodness of fit
while regulating the overall model complexity. The AIC and BIC cannot,
however, easily be applied to the models adopted here, as the use of an
informative prior distributions reduces the amount of overfitting and produces
an ``effective'' number of parameters that is smaller than the number of
variables in the parameterized model \citep{Gelman14}. We adopt the DIC, as
opposed to the AIC or BIC, because it effectively marginalizes over the
nuissance parameters and applies to the full set of posterior predictions.}}
of model parameters $p_D$ can be calculated as:
%
$$p_D = \langle D(\theta) \rangle - D(\langle \theta \rangle),$$
%
where $\langle D(\theta) \rangle$ is the mean posterior value of the deviance,
and $D(\langle \theta \rangle)$ is the deviance of the mean posterior model
parameters. We then define the deviation information criterion (DIC) as:
%
$$\mathrm{DIC} = p_D + \langle D(\theta) \rangle.$$
% 
Smaller values of  the DIC are  preferred to larger values.

Following \citet{Jeffreys61} we consider SNe with
%
$$\exp\left(\frac{\mathrm{DIC}_{\alpha2} - \mathrm{DIC}_\mathrm{flat}}{2}\right) \ge 30,$$
%
where $\mathrm{DIC}_{\alpha2}$ is the DIC for the $\alpha = 2$ prior and
$\mathrm{DIC}_\mathrm{flat}$ is the DIC for the uninformative prior, to
exhibit a very strong preference for the uninformative prior. Of the 127 SNe
in our sample, including the 7 SNe that are not considered normal SNe Ia, only
29 show a strong preference for the uninformative prior. Of these 29 SNe, 16
belong to the unreliable group. Visual inspection of these 16 confirms that
these SNe have very few detections after \tfl. In these cases the data are
extremely well fit with very small values of $\alpha$ (see e.g.,
Figure~\ref{fig:biggap_lc}). The remaining 13 SNe are at low $z$, with few, if
any, gaps in observational coverage.

Thus, the $\alpha = 2$ prior should be used to estimate \tfl\ for all but 13
SNe in our sample. For these 13, the uninformative prior provides a better
estimate of \tfl, according to the DIC. This combination of results is how we
define the distribution of \tfl\ in Paper III of this series \citep{Bulla20}.

\section{A Volume Limited Sample of Normal SNe Ia}\label{sec:volume_limited}

The ZTF sample of SNe Ia is clearly biased due to Malmquist selection effects
(see \citealt{Yao19}), and as such, the population results discussed above are
also correspondingly biased. We can, however, approximate a volume limited
subset of \textit{normal} SNe Ia. A full study of the completeness of the ZTF
SNe Ia sample is beyond the scope of this paper and will be discussed in a
future study (J.~Nordin et al., in prep.).

The selection criteria presented in Paper I removes SNe Ia from the sample if
they lack a \gztf\ detection $> 10$\,d prior to \tbmax\ \citep{Yao19}. By
construction, the intrinsically faintest normal SNe Ia in the ZTF sample have
$x_1 \approx -2$, with $M_g \approx -17$\,mag at $t \approx -10$\,d. With a
typical limiting magnitude of $g_\mathrm{ztf} \approx 20.0$\,mag during bright
time \citep{Bellm19}, the ZTF high-cadence survey should be complete to all
$x_1 \approx -2$ and brighter SNe to a distance modulus $\mu \approx 37$\,mag.
For our adopted cosmology, this distance corresponds to a redshift $z \approx
0.0585$. Thus, the 28 normal SNe Ia with $z < 0.06$ should comprise a
volume-limited subset of our sample.

For the uninformative prior, 16 of the 28 low redshift SNe have reliable model
parameters, and 15 of those 16 have known host redshifts. Using the same
procedure as \S\ref{sec:mean_rise}, we estimate a weighted mean rise time of
$\sim$18.5\,d when considering the volume complete subset ($z < 0.06$) of our
sample (see Table~\ref{tab:mean_params} for the mean values discussed here and
in the remainder of this section). For these SNe, we also find mean values of
$\sim$2.13 and $\sim$2.01 for $\alpha_g$ and $\alpha_r$, respectively.

For the $\alpha = 2$ prior, 27 of the 28 low redshift SNe have reliable model
parameters, and 24 of those 27 have known host redshifts. For this prior, we
estimate a weighted mean rise time of $\sim$21.2\,d.

If we instead use the results from the $\alpha = 2$ prior, unless the DIC
provides very strong evidence for the uninformative prior, as suggested at the
end of \S\ref{sec:dic}, then we find a mean rise time of $\sim$19.5\,d for the
volume-limited sample. It makes sense that this mean is nearly 2\,d shorter
than the mean for the $\alpha=2$ prior because the SNe for which the DIC
prefers the uninformative prior provide the tightest constraints on \trise.

Finally, if we reduce the sample to only those SNe for which the DIC prefers
the uninformative prior, of which there are only 9 normal SNe with $z < 0.06$
(all of which have known $z_\mathrm{host}$) in our entire sample, we find a
mean rise time of $18.83 \pm 0.03$. For this same subset we find mean values
of $\alpha_g$ and $\alpha_r$ of $2.14 \pm 0.03$ and $2.02 \pm 0.03$,
respectively.

Given the bias identified in \S\ref{sec:redshift_correlations}, it is not
surprising that a volume-limited sample of SNe has larger estimates for the
mean values of \trise\ and $\alpha$, when using the uninformative prior. For
the $\alpha = 2$ prior, on the other hand, the volume-limited sample produces
a very similar estimate for the mean \trise\ ($< 1\%$ difference) as the full
sample. This provides additional evidence that the adoption of a strong prior
can negate the redshift bias highlighted in \S\ref{sec:redshift_correlations}.

\section{Discussion}

\subsection{SNe Ia Rise Times}

In the analysis above, we provide multiple measurements of the rise time of SNe
Ia following the adoption of different priors. Within the literature, there are
at least a half dozen entirely different methods that have been employed to
answer precisely the same question. This naturally raises the question -- which
method is best? Which raises an important offshoot as well -- is the method
cheap to implement (i.e., does it provide reliable inference in the limit of
poor sampling or low SNR)?

\subsubsection{\texttt{SALT2} $x_1$ as a Proxy for \trise}\label{sec:x1_rise}

Estimating the rise times of SNe Ia using only observations around maximum
brightness would be an ideal approach, as the required observations are
relatively cheap. Such an approach would also maximize the sample size from
flux-limited surveys, and Figure~\ref{fig:tsquared_z_evolution} suggests this
might be feasible given the correlation between the \texttt{SALT2} $x_1$
parameter and \trise\ (measured using the $\alpha = 2$ prior).\footnote{Other
shape parameters, such as the stretch, $s$, or distance from a fiducial
template, $\Delta$, may work in place of $x_1$.} Even in the limit of only one
or a few observations on the rise, \texttt{SALT2} can still measure $x_1$
(e.g., \citealt{Scolnic18a}). Therefore, the correlation between $x_1$ and
\trise\ eliminates the need for high-cadence observations to yield early ($>
10$\,d prior to \tbmax) discoveries, enabling a more economical method to
estimate \trise\ relative to the methods described above.

For the volume-limited sample (see \S\ref{sec:volume_limited}) of normal SNe
Ia with known host galaxy redshifts and reliable model parameters, we estimate
the relation between \trise\ and $x_1$ via a maximum-likelihood linear-fit
that accounts for the uncertainties on both \trise\ and $x_1$ (see
\citealt{Hogg10}). From this fit we find
%
\begin{equation}
    t_\mathrm{rise} = (20.94 \pm 0.03) + (1.47 \pm 0.03)x_1\,\mathrm{d}.
    \label{eqn:rise_x1}
\end{equation} 
%
The residual scatter about this relation, as estimated by the sample standard
deviation, is 0.77\,d. We find the relation does not significantly change when
including SNe with unknown host-galaxy redshifts or unreliable model
parameters (though the scatter increases to $\sim$1.1\,d when including $z \ge
0.06$ SNe in the fit). Thus, if one assumes $\alpha = 2$, then \texttt{SALT2}
can be used to estimate \trise\ with a typical uncertainty of $\sim$0.8\,d.
This scatter is only slightly worse than the median uncertainty on \trise,
$\sim$0.5\,d, for individual SNe when adopting the $\alpha = 2$ prior (see
\S\ref{sec:strong_priors}). Furthermore, given that an $x_1 = 0$ SN is
supposed to represent a ``mean'' SN Ia, Equation~\ref{eqn:rise_x1} suggests
that the mean rise time of SNe Ia is $\sim$21\,d.

If we repeat the same exercise using uninformative prior rise times for the
volume limited sample, we find that the typical scatter about the linear
\trise-$x_1$ relation is $\sim$1.7\,d and $\sim$1.4\,d for the full sample and
reliable group, respectively. The rise time of a mean SN according to this
relation is $\sim$18\,d, however, we caution that some individual rise time
measurements for this prior may be underestimated as discussed in
\S\ref{sec:redshift_correlations}.

\subsubsection{Precise Estimates of \trise\ From Early Observations}

While the \trise--$x_1$ relation provides a relatively cheap method to infer
the rise time of normal SNe Ia, a significant advantage of early observations
is that they can provide far more precise estimates of \trise, especially in
the limit of high SNR. For the $\alpha = 2$ prior there are 14 SNe with a half
68\% credible region that is $< 3$\,hr. For the uninformative prior this
number drops to two SNe. In either case, these measurements provide far more
precision than possible from extrapolations based on SN Ia shape parameters
(such as $x_1$).

While the methods adopted in this paper provide higher precision, it is
impossible that they are both accurate. The median difference in the inferred
\trise\ from the $\alpha = 2$ and uninformative priors for individual SNe is
4.7\,d. Even the volume-limited subset of SNe with reliable model parameters
from the uninformative prior (15 total SNe) have a median difference of 2.8\,d
in \trise\ for the two priors. The systematic effect identified in
\S\ref{sec:redshift_correlations} suggests that the uninformative prior does
not provide accurate estimates of \trise\ for higher-$z$ SNe. The
$\alpha = 2$ prior, on the other hand, explicitly assumes that there is no
change in the early optical color of SNe Ia. Many SNe with early observations
clearly invalidate this particular assumption, raising the possibility that
neither method is accurate.

% For \trise, empirical evidence suggests that the uninformative prior
% underestimates the true rise time of SNe, meaning the $\alpha = 2$ prior may
% be more accurate (though again we caution that these estimates may also be
% inaccurate).
A SN cannot be detected until it has exploded, and thus the epoch of discovery
provides a lower limit on \trise. Between PTF/iPTF \citep{Papadogiannakis19}
and ZTF \citep{Yao19}, there are $\sim$20 SNe Ia that are detected at least
18\,d before \tbmax, with a few detections as early as 21\,d before \tbmax. If
the uninformative prior is accurate, then each of these SNe would represent an
incredibly lucky set of circumstances: (i) they each have longer rise times
than average ($\sim$18\,d; see \S\ref{sec:volume_limited}), and (ii) they were
all discovered more or less immediately after \tfl. A more probable
explanation is that the mean rise time is $> 18$\,d, in which case the $\alpha
= 2$ prior may provide a more accurate inference of the rise time (though
again we caution that these estimates may also be inaccurate). The fact that
each of the 4 SNe with $z < 0.03$, which should have the least biased rise
time estimates (see \S\ref{sec:redshift_correlations}), have \trise$ > 18$\,d,
further supports this claim.

\subsubsection{Comparison to the Literature}

Several studies in the literature have attempted to measure the mean rise time
of SNe Ia. Here we compare our work to previous results. This exercise is
somewhat fraught with difficulty, in the sense that each study incorporates
slight differences in implementation, which, in turn, makes comparisons
challenging. Furthermore, these studies are typically conducted with different
filter sets and over a wide range of redshifts, which may introduce biases
that are difficult to quantify across studies (as discussed above
$K$-corrections are highly uncertain at very early epochs). Finally, the
quality of the data in each of these studies is vastly different. For example,
in \citet{Riess99a} there are only 6 SNe (and 10 total $B$-band observations)
observed at phases $\le -15$\,d, while our study includes 31 SNe
\textit{discovered $> 15$\,d before \tbmax} \citep{Yao19}. As we proceed with
our cross-study comparison, we exclude rise time estimates for individual SNe
and instead focus on studies with relatively large samples ($\gtrsim 10$
normal SNe Ia).

As previously outlined, there are broadly two different methods to measure the
mean rise time of SNe Ia. The first uses the well established
luminosity-decline relation for SNe Ia \citep{Phillips93} to ``shape correct''
the SN light curves prior to fitting for the rise time. Thus, individual light
curves are stretched by some empirically measured factor, and the mean rise
time represents a normal SN Ia \textit{after shape correction} (e.g.,
\citealt{Riess99a,Aldering00,Conley06,Hayden10,Ganeshalingam11}). The second
method measures the rise time of each SN Ia within a sample, and then takes
the mean of this distribution. These two methods are not equivalent, and
therefore are likely to produce different results. If, for instance, a
flux-limited survey finds more high-luminosity, slower-declining SNe than
low-luminosity, faster-declining events, then the population mean will produce
longer rise times than the shape-corrected mean.

Using different data sets obtained in very different redshift regimes,
\citet{Riess99a}, \citet{Aldering00}, and \citet{Conley06} estimate consistent
values of the shape-corrected mean \trise$ \approx 19.5$\,d. Each of these
studies fixes $\alpha = 2$ when fitting for the rise time. This estimate is
roughly half way in between our estimates of the mean rise time from the
uninformative prior and the $\alpha = 2$ prior. Later studies by
\citet{Hayden10} and \citet{Ganeshalingam11}, also provide estimates of the
mean shape-corrected rise time and find smaller values of $\sim$17.5--18.0\,d.
As noted by \citeauthor{Hayden10}, these methods are highly dependent on the
template light curve used to stretch the individual SNe, and differences in
the templates used by these studies may explain the dissensus between their
findings.

The approach employed in \citet{Zheng17a}, \citet{Papadogiannakis19}, and
\citet{Firth15} is more similar to the one adopted here, as each of these
studies estimates the rise time for many individual SNe and then calculates
the population mean. If the samples differ between any of these studies, and
aside from 11 SNe that are included in both \citet{Papadogiannakis19} and
\citet{Firth15} there is no overlap between any of those studies or this one,
then it should be expected that the population mean rise time estimates will
differ. Furthermore, the \trise\ estimates in \citet{Papadogiannakis19} and
\citet{Firth15} are not relative to \tbmax, and it is known that the rise time
increases as one progressively moves to redder wavelengths (e.g.,
\citealt{Ganeshalingam11}). Taken together, this confluence of factors makes
it difficult to compare results between these studies, which we nevertheless
do below.

In \citet{Zheng17}, a semi-analytical, six parameter, broken power-law model
is introduced to describe the optical evolution of SNe Ia. This model has a
distinct advantage over the methods employed here in that an artificial cutoff
does not need to be applied in flux-space (see \S\ref{sec:model}), though a
post-peak cut must be applied as the model cannot reproduce the evolution of
SNe into the nebular phase. A downside of this approach is that there are
large degeneracies between the different model parameters, meaning it is
difficult to find numerically stable solutions without fixing individual
parameters to a single value \citep{Zheng17a}. For a sample of 56
well-observed low-$z$ SNe this method produces a mean rise time of 16.0\,d
\citep{Zheng17a}, while the same technique applied to SNe Ia from PTF/iPTF
finds a mean rise time of 16.8\,d \citep{Papadogiannakis19}. These estimates
are considerably lower than the ones presented here, and are almost certainly
underestimates of the true mean rise time based on the large number of SNe
with detections $>$16\,d before peak \citep{Papadogiannakis19, Yao19}. Indeed,
inspection of Figure~1 in \citet{Zheng17a} shows that the six-parameter model
underestimates the flux at the very earliest epochs and underestimates \trise\
as a result.

The closest comparison to the methods used in this study can be drawn from
\citet{Firth15}. Using a sample of 18 SNe discovered by PTF and the La Silla
Quest (LSQ) survey, \citeauthor{Firth15} fit a model similar to
Equation~\ref{eqn:flux_model}, in that \tfl\ and $\alpha$ are allowed to
simultaneously vary. From these fits, they estimate a mean population \trise$
= 18.98 \pm 0.54$\,d, which is consistent with our estimate of the rise time
for the volume-complete $z_\mathrm{host}$ sample, $\sim$18.5\,d. Contrary to
this study, they find shorter rise times, and a mean of $\sim$17.9\,d, when
fixing $\alpha = 2$ (this is likely explained by their adopted fit procedure,
see \S\ref{sec:fireball_discussion}).

\subsection{The Expanding Fireball Model}\label{sec:fireball_discussion}

The expanding fireball model (see \S\ref{sec:model}) is remarkable in its
simplicity. The two underlying assumptions of the expanding fireball model,
that the photospheric velocity and temperature of the ejecta are
$\sim$constant during the early evolution of the SN, are clearly
over-simplifications (\citealt{Parrent12} shows that the photospheric velocity
declines by at least 33\% in the $\sim$5\,d after explosion). Despite these
simplifications, numerous studies have found that $\alpha$ is consistent with
2 (e.g.,
\citealt{Conley06,Hayden10,Ganeshalingam11,Gonzalez-Gaitan12,Zheng17a}).

Based on the volume-limited subset of normal SNe Ia with reliable model
parameters, we find a population mean $\alpha_r = 2.01 \pm 0.03$, which is
consistent with the expanding fireball simplification. For $\alpha_g$, on the
other hand, we find a population mean of $2.13 \pm 0.03$, which is only
marginally consistent with 2. As outlined above, it stands to reason that
$\alpha$ would not equal 2 in every optical filter, as this would suggest SNe
do not change colors shortly after explosion.

Furthermore, there are individual normal SNe Ia for which the expanding
fireball model does not apply. This has been clearly shown for several SNe (e.g., \citealt{Zheng13,Zheng14,Goobar15,Miller18,Shappee19,Dimitriadis19}),
and within this study that is clearly the case for several of the lowest
redshift SNe, with hence the highest SNR detections, in our sample:
% 
\begin{itemize}
\item ZTF18aasdted (SN\,2018big; $z \approx 0.018$, $\alpha_r \approx 1.4$),
\item ZTF18abauprj (SN\,2018cnw; $z \approx 0.024$, $\alpha_r \approx 2.2$),
\item ZTF18abcflnz (SN\,2018cuw; $z \approx 0.027$, $\alpha_r \approx 2.4$),
\item ZTF18abfhryc (SN\,2018dhw; $z \approx 0.032$, $\alpha_r \approx 3.3$), 
\item ZTF18abuqugw\;(SN\,2018geo; $z\,\approx\,0.031$, $\alpha_r\approx2.7$).
\end{itemize}
% 
 In each of these cases the DIC clearly prefers $\alpha \neq 2$.

Given that many of the very best-observed, low-redshift SNe are incompatible
with $\alpha = 2$, and that the mean $\alpha_g > 2$ in the ZTF sample, it stands
to reason that the expanding fireball model does not adequately reproduce the
observed diversity of SNe Ia at early phases. Nevertheless, according to the
DIC, $\alpha=2$ provides a reasonable proxy for the early evolution of the
majority of normal SNe\,Ia (at the quality of ZTF high-cadence observations).
This is either telling us that individual SNe exhibiting significant
departures from $\alpha = 2$ are atypical (an interpretation adopted in
\citealt{Hosseinzadeh17,Miller18,Dimitriadis19} and elsewhere), or, that for
the vast majority of SNe the observations are not of a high enough quality to
conclusively show $\alpha \neq 2$. Distinguishing between these two
possibilities requires larger volume-limited samples.

Moving forward, it may be that the most appropriate prior for fitting the
early evolution of SNe Ia is to adopt a Gaussian centered at 2 for $\alpha$ in
the redder filters, while also placing a prior on the difference in $\alpha$
across different filters (for ZTF $\alpha_r - \alpha_g \approx -0.18$ based on
\S\ref{sec:colors}). More observations, especially of low-$z$ SNe, and
testing, are needed to confirm whether or not such priors are in fact
appropriate.

Finally, we note that the analysis in \citet{Firth15} finds a mean value of
$\alpha = 2.44 \pm 0.13$, which is not consistent with 2. This result can be
understood in the context of the \citet{Firth15} fitting procedure, whereby an
initial estimate of \tfl\ is made by fixing $\alpha = 2$. Only observations
obtained 2 days before and after this initial \tfl\ estimate are included in
the final model fit (i.e., the entire baseline of non-detections is not used,
as is done in this study). Truncating the baseline biases the model to longer
rise times (as is observed in \citealt{Firth15}), and, as shown in
Figures~\ref{fig:corner_good}, \ref{fig:corner_median}, and
\ref{fig:corner_bad}, longer rise times require larger values of $\alpha$ when
adopting a simple power-law model (as is done here and in \citet{Firth15}).

The reason it is critical to test the expanding fireball model is that robust
measurements of $\alpha$ can distinguish between different explosions
scenarios. For example, the delayed-detonation models presented in
\citet{Blondin13}, which provide a good match to SNe Ia at maximum light,
systematically over estimate the power-law index at early times (with typical
values of $\alpha \approx 7$, see Figure~1 in \citealt{Dessart14}). This led
\citet{Dessart14} to alternatively consider pulsational-delayed detonation
models, which do result in a smaller power-law index ($\alpha \approx 3$),
though those results are still incompatible with what we find
here.\footnote{The range of $\alpha$ values reported in \citet{Dessart14} is
fit to the first $\sim$3\,d after explosion. Fitting all observations with
$f_\mathrm{obs} \leq 0.4 f_\mathrm{max}$, as is done in this study, would
reduce the inferred values of $\alpha$ in \citet{Dessart14}, possibly bringing
them in line with those found here.} In \citet{Noebauer17}, the early
evolution of various explosion models does not follow an exact power-law. They
find an almost power-law evolution for pure deflagration models, which may
explain the origin of 02cx-like SNe. For pure deflagrations,
\citet{Noebauer17} find $\alpha < 2$, which qualitatively agrees with our
results for ZTF18abclfee (SN\,2018cxk), where $\alpha_r \approx 1$ (see
\S\ref{sec:rare}). The models presented in \citet{Magee20}, which examine the
evolution of SNe with different $^{56}$Ni distributions, provide good
qualitative agreement to what we find for ZTF SNe. \citet{Magee20} find that
the rising power-law index is larger in the $B$-band than the $R$-band
(similar to what we see in \gztf\ and \rztf), and that the mean value of these
distributions is $\sim$2. As suggested in \citet{Magee20}, it may be the case
that the vast majority of the differences observed in the early ZTF light
curves can be explained via variations in the $^{56}$Ni mixing in the SN
ejecta. Future modeling will test this possibility.


\section{Conclusions}

In this paper we have presented an analysis of the initial evolution and rise
times of 127 ZTF-discovered SNe Ia with early observations (see
\citealt{Yao19} for details on how the sample was selected). These SNe were
observed as part of the ZTF high-cadence extragalactic experiment, which
obtained 3 \gztf\ and 3 \rztf\ observations every night the telescope was
open. A key distinction of this data set relative to many previous studies is
the large number of observations taken prior to the epoch of discovery, which
meaningfully constrains the behavior of the SN at very early times (see
\ref{sec:pre_explosion}). The uniformity, size and observational duty cycle of
this data set are truly unique, making this sample of ZTF SNe the premiere
data set for studying the early evolution of thermonuclear SNe.

We model the emission from these SNe as a power-law in time $t$, whereby the
flux $f \propto (t - t_\mathrm{fl})^\alpha$, where \tfl\ is the time of first
light, and $\alpha$ is the power-law index. By simultaneously fitting
observations in the \gztf\ and \rztf\ filters, we are able to place stronger
constraints on \tfl\ than would be possible with observations in a single
filter. While many previous studies have fixed $\alpha = 2$, following the
simple expanding fireball model (e.g., \citealt{Riess99a}), we have instead
allowed $\alpha$ to vary, as there are recent examples of SNe Ia where
$\alpha$ clearly is not equal to 2 (e.g.,
\citealt{Zheng13,Zheng14,Goobar15,Miller18,Shappee19,Dimitriadis19}). While
the population mean value of $\alpha$ tends towards 2, there are several
individual SNe featuring an early evolution that deviates from an $\alpha = 2$
power-law, justifying our model parameterization.

As might be expected, we find that our ability to constrain the model
parameters is highly dependent on the quality of the data. SNe Ia at low
redshifts that lack significant gaps in observational coverage are better
constrained than their high-redshift counterparts or events will large
temporal gaps. We identify those SNe with reliable model parameters under the
reasonable assumption that models of the initial flux evolution should
over-estimate the flux at peak brightness. Following this procedure we find
that 51 of the SNe have reliable model parameters. We focus our analysis on
these events.

For the subset of normal SNe with reliable model parameters we estimate a
population mean \trise\;$\approx 18.0$\,d, with a sample standard deviation of
$\sim$1.6\,d. For individual SNe, the range of rise times extends from
$\sim$14--22\,d. We have additionally identified a systematic in the parameter
estimation for models that simultaneously vary \tfl\ and $\alpha$. Namely, for
flux-limited surveys, the model constraints on \trise\ will be systematically
underestimated for the higher redshift SNe in the sample. If we restrict the
sample to a volume-limited subset of SNe ($z < 0.06$), where this bias may
still be present but probably less prevalent, we estimate a mean population
rise time of $\sim$18.5\,d.

Normal SNe Ia have a population mean $\alpha_g \approx 2.1$ and a population
mean $\alpha_r \approx 2.0$, with a population standard deviation $\sim$0.5
for both parameters. While the mean value for our sample of SNe tends towards
2, as would be expected in the expanding fireball model, we observe a range in
$\alpha$ extending from $\sim$1.0--3.5. For both \trise\ and $\alpha$, there
is no single value that is consistent with all the SNe in our sample.
Interestingly, however, we do find that nearly all SNe are consistent with a
single value of $\alpha_r - \alpha_g$, which describes the initial \gztf$ -
$\rztf\ color evolution of SNe Ia. The data show a mean value of $\alpha_r -
\alpha_g \approx -0.18$, meaning the optical colors of most SNe Ia evolve to
the blue with comparable magnitudes over a similar timescale. This could be a
sign that the degree of $^{56}$Ni-mixing in the SN ejecta is very similar for
the majority of SNe Ia (e.g., \citealt{Piro16,Magee18,Magee20}).

We find that the rise time is correlated with the light curve shape of the SN,
in the sense that high-luminosity, slowly-declining SNe have longer rise
times. This finding is consistent with many previous studies.

Given the large number of SNe with unreliable model parameters, and the
observed bias in the measurement of \trise\ for high-$z$ SNe, we also consider
how the model parameter estimates change with strong priors. In particular, we
adopt $\alpha_g = \alpha_r = 2$, enforcing the expanding fireball hypothesis
on the data. Strictly speaking, this prior means that the early colors of SNe
Ia do not change, which we empirically know is not the case. Nevertheless, a
$\sim$constant temperature is one of the assumptions of the fireball model,
and, thus we proceed.

Under the $\alpha = 2$ prior, we find that far more SNe have reliable \trise\
estimates. For the typical SN in our sample, fixing $\alpha = 2$ results in an
increase in \trise\ by a few days. We estimate a population mean \trise\;$
\approx 21.0$\,d when adopting the $\alpha = 2$ prior. One consequence of
adopting this prior is that it significantly reduces the previously observed
bias where high-$z$ SNe are inferred to have shorter rise times. The use of
this prior also reduces the scatter in the $x_1$--\trise\ relation, and we
find that with \texttt{SALT2}, via the measurement of $x_1$, it is possible to
estimate \trise\ with a typical scatter of $\sim$0.77\,d, even if there are no
early time observations available. We also find that, for the vast majority of
the SNe in our sample, all but 13 events, there is, at best, only weak
evidence that the $\alpha \ne 2$ model is preferred to a model with $\alpha_g
= \alpha_r = 2$ according to the DIC.

While we have primarily focused on the properties and evolution of normal SNe
Ia, there are 7 SNe in our sample that cannot be categorized as normal (see
\citealt{Yao19}). As in previous studies, we find that the rise times of
Ia-CSM SNe and SC explosions are longer than those of normal SNe Ia. We
highlight our observations of ZTF18abclfee (SN\,2018cxk), an 02cx-like SN with
exquisite observational coverage in the time before explosion. We estimate the
\tfl\ to within $\sim$8\,hr for ZTF18abclfee, making our measurement the most
precise estimate of \trise\ for any 02cx-like SN to date. ZTF18abclfee took
$\sim$10\,d to reach peak brightness, roughly 5\,d less than SN\,2005hk,
another 02cx-like event with a well-constrained rise time.

This study has important lessons for future efforts to characterize the rise
times of SNe Ia. We have found that for all but the best-observed, highest SNR
events, a generic power-law model where $\alpha$ is allowed to vary does not
place meaningful constraints on the \trise, or worse, in the case of high-$z$
events, it produces a biased estimate. If this were the end of the story, it
would be particularly bad news for LSST, which will typically have several day
gaps in its observational cadence \citep{Ivezic08}. With only
Equation~\ref{eqn:flux_model} at our disposal, we would rarely be able to
infer \trise\ for LSST SNe. We have also shown, however, that in the limit of
low-quality data, the application of a prior to the model can significantly
improve the final inference. Our current challenge is to develop an
empirically motivated prior for the model parameters. This provides a strong
justification for the concurrent operation of LSST and small-aperture,
high-cadence experiments, such as ZTF and the planned ZTF-II. These smaller,
more focused, missions can provide exquisite observations of a select handful
of SNe that can be used to drive the priors in our inference. While there have
been thousands of SNe Ia studied to date (e.g., \citealt{Jones17}), indeed
more examples are still needed: there are only 4 normal, $z < 0.03$ SNe Ia in
our sample, and these 4 are nearly as valuable as the remainder of the sample
for establishing the diversity of SNe Ia. As we improve our understanding of
these priors, and combine that knowledge with the hitherto unimaginable
statistical samples from LSST ($\sim$millions of SNe), we will definitively
understand the early evolution and rise time distribution of Type Ia SNe.

\acknowledgements

We thank M.~Magee for sharing details about the rise times of SNe Ia models.
A.A.M.~would like to thank E.~A.~Chase, M.~Zevin, and C.~P.~L.~Berry for
useful discussions on KDEs and PDFs. We also appreciate D.~Goldstein's
suggestions regarding \texttt{SALT2} as a proxy for rise time. Y.~Yang,
J.~Nordin, R.~Biswas, and J.~Sollerman provided detailed comments on an early
draft that improved this manuscript.

A.A.M.~is funded by the Large Synoptic Survey Telescope Corporation, the
Brinson Foundation, and the Moore Foundation in support of the LSSTC Data
Science Fellowship Program; he also receives support as a CIERA Fellow by the
CIERA Postdoctoral Fellowship Program (Center for Interdisciplinary
Exploration and Research in Astrophysics, Northwestern University). Y.~Y.,
U.~C.~F., and S.~R.~K.~thank the Heising-Simons Foundation for supporting ZTF
research (\#2018-0907). This research was supported in part through the
computational resources and staff contributions provided for the Quest high
performance computing facility at Northwestern University which is jointly
supported by the Office of the Provost, the Office for Research, and
Northwestern University Information Technology. This work was supported in
part by the GROWTH project funded by the National Science Foundation under
Grant No.~1545949.

This work is based on observations obtained with the Samuel Oschin Telescope
48-inch and the 60-inch Telescope at the Palomar Observatory as part of the
Zwicky Transient Facility project. ZTF is supported by the National Science
Foundation under Grant No. AST-1440341 and a collaboration including Caltech,
IPAC, the Weizmann Institute for Science, the Oskar Klein Center at Stockholm
University, the University of Maryland, the University of Washington,
Deutsches Elektronen-Synchrotron and Humboldt University, Los Alamos National
Laboratories, the TANGO Consortium of Taiwan, the University of Wisconsin at
Milwaukee, and Lawrence Berkeley National Laboratories. Operations are
conducted by COO, IPAC, and UW.

\software{
          \texttt{astropy} \citep{Astropy-Collaboration13},
          \texttt{scipy} \citep{2020SciPy-NMeth}, 
          \texttt{matplotlib} \citep{Hunter07},
          \texttt{pandas} \citep{McKinney10},
          \texttt{emcee} \citep{Foreman-Mackey13},
          \texttt{corner} \citep{Foreman-Mackey16},
          \texttt{SALT2} \citep{Guy07},
          \texttt{sncosmo} \citep{Barbary16},
          \texttt{statsmodels} \citep{Seabold10}
          }

%% For this sample we use BibTeX plus aasjournals.bst to generate the
%% the bibliography. The sample63.bib file was populated from ADS. To
%% get the citations to show in the compiled file do the following:
%%
%% pdflatex sample63.tex
%% bibtext sample63
%% pdflatex sample63.tex
%% pdflatex sample63.tex

\appendix

\section{Updated Priors Following the Change of Variables}\label{sec:prior}

As mentioned in \S\ref{sec:model}, there is a strong degeneracy in the
posterior estimates of $A$ and $\alpha$. This degeneracy can be removed under
the change of variables from ($A, \alpha$) to ($A^\prime, \alpha^\prime$),
where $A^\prime = A 10^\alpha$ and $\alpha^\prime = \alpha$. From the Jacobian
of this transformation, we find
%%
$$P(A^\prime, \alpha^\prime) = 10^{-\alpha^\prime} P(A,\alpha).$$
%%
The change in variables should not affect the prior probability, therefore,
%%
\begin{equation} 
    P(A^\prime, \alpha^\prime) = 10^{-\alpha^\prime} P(A^\prime
10^{-\alpha^\prime},\alpha^\prime), 
\label{eqn:change_prior} 
\end{equation}
%%
which can be satisfied by:
%%
\begin{equation}
    P(A^\prime, \alpha^\prime) \propto {A^\prime}^{-1} 10^{-\alpha^\prime}.
\label{eqn:transform}
\end{equation}
%%
While Equation~\ref{eqn:change_prior} is also satisfied by $P(A^\prime,
\alpha^\prime) \propto {A^\prime}^{-1}$, adopting this as the joint prior on
($A^\prime, \alpha^\prime$) does not remove the degeneracy between the
parameters as $A^\prime$ absorbs the multiplicative factor of $10^\alpha$,
effectively reducing the problem to be the same as it was before the change of
variables. Thus, as listed in Table~\ref{tab:priors}, we adopt
Equation~\ref{eqn:transform} as the prior on the transformed variables, which
we find breaks the degeneracy (see Figures\ref{fig:corner_good},
\ref{fig:corner_median}, and \ref{fig:corner_bad}).

\section{Quality Assurance}\label{sec:qa}

As noted in \S\ref{sec:model}, the MCMC model converges for all but one ZTF
SNe within the sample. However, visual inspection of both the corner plots and
individual draws from the posterior quickly reveals that for some SNe the data
do not provide strong constraints on the model parameters (see Figure~\ref{fig:corner_bad}). In the most extreme cases, as shown in
Figure~\ref{fig:biggap_lc}, large gaps in the observations make it nearly
impossible to constrain the model parameters. For these cases, the model
posteriors are essentially identical to the priors (there is always a weak
constraint on \tfl\ from epochs where the SN is not detected).

\begin{figure}
    \centering
    \includegraphics[width=3.4in]{./figures/ZTF18aaqffyp_model_lc.pdf}
    %
    \caption{Same as the bottom panel of Figure~\ref{fig:corner_good} for
    ZTF18aaqffyp (SN\,2018bhr), a SN with observations that place very weak
    constraints on \tfl. The marginalized posteriors for $A^\prime$ and
    $\alpha$ are essentially identical to the priors. Posteriors with little
    information beyond the prior are typical of SNe with significant
    observational gaps.}
    %
    \label{fig:biggap_lc}
\end{figure}

To identify SNe with poor observational coverage, or unusual structure in the
posterior, we visually examine the light curves and corner plots for each of
the 127 SNe in our sample. We flag SNe where the model significantly
underestimates the flux near \tbmax\ (similar to what is shown in
Figure~\ref{fig:biggap_lc}), as this is a good indicator that the model has
poor predictive value. By definition the light curve derivative is zero at
maximum light, and the relative change in brightness constantly slows down in
the $\sim$week leading up to maximum light. Therefore, models of the early
emission should greatly over-predict the flux at maximum, which is why we
adopt this criterion for flagging SNe with poorly constrained model
parameters.

\begin{figure}
    \centering
    \includegraphics[width=3.5in]{./figures/final_sample.pdf}
    %
    \caption{Scatter plot showing the distribution of the 127 ZTF SNe Ia in
    the $\mathrm{GM}(N_\mathrm{det})$--$\mathrm{CR}_{90}$ plane. Models with
    flagged posterior parameters are shown as orange circles, while those
    that are not flagged are shown via $+$ symbols. The dashed line shows the
    adopted separation threshold for identifying reliable model fits (above
    the line), and unreliable model fits. FP and FN (see text) SNe are
    circled. }
    %
    \label{fig:flagged_sn}
\end{figure}

Numerically, the visually flagged SNe can, for the most part, be identified
by a combination of two criteria: the 90\% credible region on \tfl,
$\mathrm{CR}_{90}$, and the number of nights on which the SN was detected.
Rather than providing a threshold for detection (e.g., $3\sigma$, $5\sigma$,
etc.), we count all nights with $f_\mathrm{mean} \le 0.4 f_\mathrm{max}$
after the median marginalized posterior value of \tfl\ with observations in
either the \gztf, \rztf, or both filters, $N_{g, \mathrm{det}}$, $N_{r,
\mathrm{det}}$, and $N_{gr, \mathrm{det}}$, respectively. We take the
geometric mean of these three numbers to derive the ``average'' number of
nights on which the SN was detected, $\mathrm{GM}(N_\mathrm{det})$. A scatter
plot showing $\mathrm{GM}(N_\mathrm{det})$ vs.\ $\mathrm{CR}_{90}$ is shown
in Figure~\ref{fig:flagged_sn}. Visually flagged SNe are shown as orange
circles, while $+$ symbols show those that were not flagged.

The visual inspection procedure described above is not fully reproducible
(visual inspection is by its very nature subjective). Therefore, we aim to
separate the SNe into two classes (reliable and unreliable) via an automated,
systematic procedure. Treating the visually flagged sources as the negative
class, we view false positives (i.e., visually flagged SNe that are included
in the final population analysis) particularly harmful. Therefore, we adopt
%
$$ \mathrm{GM}(N_\mathrm{det}) \ge 1.9\,\mathrm{CR}_{90} + 1.65,$$
%
as the classification threshold for reliable model fits (as shown via the
dashed line in Figure~\ref{fig:flagged_sn}). This threshold retains 50 true
positives (TP; visually good models included in the final sample) with only a
single false positive (FP; visually flagged SNe in the final sample). This
choice does result in 20 false negatives (FN; visually good models
\textit{excluded} from the final sample), while all remaining flagged SNe are
true negatives (TN). Further scrutiny of the FN reveals several light curves
with significant observational gaps, which, as discussed above, makes it
difficult to place strong constraints on the model parameters. Ultimately, our
two-step procedure identifies 51 SNe as reliable, while 76 are excluded from
the final population analysis due to their unreliable constraints on the model
parameters.

\section{Rare and Unusual Thermonuclear SNe}\label{sec:rare}

\edit1{In \citet{Yao19}, we identified 6 peculiar SNe Ia, which were
classified as either SN\,2002cx-like (hereafter 02cx-like or SN\,Iax),
super-Chandrasekhar (SC) explosions, or SNe Ia interacting with their
circumstellar medium (CSM), known as SN Ia-CSM. For this study we have also
excluded ZTF18abdmgab (SN\,2018lph), a 1986G-like SN that would not typically
be included in a sample used for cosmological studies. Here we summarize the
early evolution of these events.}

\edit1{For ZTF18abclfee (SN\,2018cxk), an 02cx-like SN at $z = 0.029$, we
obtained an exquisite sequence of observations in the time before explosion,
as shown in Figure~\ref{fig:02cx}. According to the DIC, $\alpha \ne 2$ is
decisively preferred for this SN. For ZTF18abclfee, we estimate \trise$ =
10.01 \pm^{0.40}_{0.33}$\,d\footnote{Rise times for the unusual SNe discussed
in this appendix are measured relative to $T_{g,\mathrm{max}}$ as
\texttt{SALT2} does not provide reliable estimates of \tbmax\ for non-normal
SNe\,Ia.} (the uncertainties represent the 90\% credible region). This is the
most precise measurement of the rise time of an 02cx-like SN to date. The only
other 02cx-like event with good limits on the rise with deep upper limits is
SN\,2005hk \citep{Phillips07}. SN\,2005hk has a substantially longer rise time
($\sim$15\,d; \citealt{Phillips07}) than ZTF18abclfee, which is not surprising
given that ZTF18abclfee is less luminous and declines more rapidly than
SN\,2005hk \citep{Miller17a,Yao19}. ZTF18abclfee also exhibits a nearly linear
early rise with $\alpha_g = 0.95 \pm^{0.32}_{0.19}$ and $\alpha_r = 0.98
\pm^{0.23}_{0.15}$.}

\begin{figure}
    \centering
    \includegraphics[width=3.4in]{./figures/ZTF18abclfee_model_lc.pdf}
    %
    \caption{Same as the bottom panel of Figure~\ref{fig:corner_good} for
    ZTF18abclfee (SN\,2018cxk), an 02cx-like SN with strong constraints on
    \tfl, and a short rise time ($\sim$10\,d). ZTF18abclfee has the tightest
    constraints on \trise\ of all 02cx-like SNe observed to date.}
    %
    \label{fig:02cx}
\end{figure}

\edit1{ZTF18aaykjei (SN\,2018crl), a Ia-CSM SN with \trise$ = 22.8
\pm^{2.0}_{1.8}$\,d and $26.3 \pm 1$\,d for the uninformative and $\alpha = 2$
priors, respectively, has a significantly longer rise than the normal SNe in
this study. \citet{Silverman13} points out that Ia-CSM have exceptionally long
rise times, and \citet{Firth15} measure \trise$ > 30$\,d for two of the SNe in
the \citet{Silverman13} sample. We also note that the \rztf\ peak of
ZTF18aaykjei occurs at least one week after the \gztf\ peak, as has been seen
in other Ia-CSM SNe \citep{aldering05gj,prieto05gj}.}

\edit1{There are two SC SNe Ia (ZTF18abdpvnd/SN\,2018dvf and
ZTF18abhpgje/SN\,2018eul) and two candidate SC SNe (ZTF18aawpcel/SN\,2018cir
and ZTF18abddmrf/SN\,2018dsx) identified in \citet{Yao19}. Each of these
events exhibits a long rise, $\gtrsim 20$\,d and $\gtrsim 25$\,d for the
uninformative and $\alpha=2$ priors, respectively, as previously seen in other
SC events (e.g., \citealt{Scalzo10,Silverman11}). We note that with the
exception of ZTF18abdpvnd ($z = 0.05$), these events are detected at high
redshift ($z \gtrsim 0.15$), and as a result the constraints on the individual
rise time measurements are relatively weak.}

\edit1{Finally, for ZTF18abdmgab (SN\,2018lph), the 86G-like SN identified in
\citet{Yao19}, we cannot place strong constraints on the rise time due to a
significant gap in the observations around \tfl.}


\section{Systematics}\label{sec:systematics}

\subsection{Definition of ``Early'' for Model Fitting}\label{sec:flux_cut}

In \S\ref{sec:model} we highlighted that there is no single agreed upon
definition of which SN Ia observations are best for modeling the early
evolution of SNe Ia. Throughout this study we have adopted a threshold,
$f_\mathrm{thresh}$, relative to the maximum observed flux, $f_\mathrm{max}$,
whereby we define all observations less than $f_\mathrm{thresh} = 0.4$ the
maximum in each filter ($f_\mathrm{obs} \leq 0.4f_\mathrm{max}$) as the early
portion of the light curve. As noted in \S\ref{sec:model}, setting
$f_\mathrm{thresh} = 0.4$, is arbitrary (although consistent with some
previous studies). Here we examine the effect of this particular choice if we
had instead adopted $f_\mathrm{thresh} = 0.25$, 0.30, 0.35, 0.45, or 0.50 for
the fitting procedure in \S\ref{sec:model}.

\begin{figure}[ht]
    \centering
    \includegraphics[width=3.4in]{./figures/flux_frac.pdf}
    %
    \caption{Evolution of the inferred values of \tfl\ and $\alpha_g$ as
    $f_\mathrm{thresh}$ is increased from 0.25 to 0.5. Only SNe with
    consistent model parameters, that nevertheless show evidence for
    increasing or decreasing with $f_\mathrm{thresh}$, are shown (see text for
    a definition of consistent, increasing, and decreasing). Thin green lines
    show SNe where \trise\ or $\alpha_g$ increases as more observations are
    included in the model, while thick purple lines show SNe for which these
    values decline. We find that for the vast majority of SNe, as additional
    observations are included in the model fit, both \trise\ and $\alpha_g$
    increase.}
    %
    \label{fig:flux_frac}
\end{figure}

There are 12 SNe for which the MCMC chains did not converge for one or more of
the alternative flux thresholds. They are excluded from the analysis below.
For the remaining 115 SNe in our sample, we consider the model parameters to
be consistent if the marginalized, 1-dimensional 90\% credible regions for the
three parameters that we care about, \tfl, $\alpha_g$, and $\alpha_r$, overlap
with the estimates when $f_\mathrm{thresh} = 0.40$.\footnote{Given the strong
correlation between $\alpha_g$ and $\alpha_r$ (see
\S\ref{sec:alpha_correlation}), we discuss only $\alpha_g$ below.} This
definition identifies substantial differences in the final model parameters
while varying $f_\mathrm{thresh}$ over a reasonable range. Of the 115 SNe with
converged chains, we find that 98 ($> 85\%$ of the sample) have marginalized,
1-D posterior credible regions consistent with the results for
$f_\mathrm{thresh} = 0.40$, independent of the adopted value of
$f_\mathrm{thresh}$. 15 of the 17 SNe that do not have consistent \tfl,
$\alpha_g$, or $\alpha_r$ estimates feature gaps in observational coverage,
which is the likely reason for the inconsistency. As $f_\mathrm{thresh}$
increases from 0.25--0.5, the information content dramatically changes before
and after a gap leading to significantly different parameter estimates.

If we alternatively consider the results to be consistent only if the 68\%
credible regions agree with the $f_\mathrm{thresh} = 0.40$ results, then only
64 SNe have consistent parameters as $f_\mathrm{thresh}$ varies. This suggests
that while the results are largely consistent, the central mass of the
posterior density is affected by which data are, and are not, included in the
model fit. In Figure~\ref{fig:flux_frac}, we show how the estimates of \tfl\
and $\alpha_g$ change as a function of $f_\mathrm{thresh}$ for SNe with
consistent model parameters. Note that, by construction, the 90\% credible
regions for each SN overlap at every value of $f_\mathrm{thresh}$, and thus,
for clarity, we omit error bars.

To identify trends with $f_\mathrm{thresh}$, we define SNe with both
$\alpha_g(f_\mathrm{thresh} = 0.5)$ and $\alpha_g(f_\mathrm{thresh} = 0.45)$
greater than both $\alpha_g(f_\mathrm{thresh} = 0.25)$ and
$\alpha_g(f_\mathrm{thresh} = 0.3)$ to show evidence for $\alpha_g$ increasing
with $f_\mathrm{thresh}$. We define $\alpha_g$ as decreasing in cases where
the opposite is true. 75 of the 98 SNe with consistent model parameters show
evidence for $\alpha_g$ increasing with $f_\mathrm{thresh}$, while only 7 show
evidence for a decline. Using a similar definition for \trise\ (note that
decreasing \tfl\ corresponds to increasing \trise), we find that in 68 SNe
\trise\ increases with $f_\mathrm{thresh}$, while in 16 SNe \trise\ decreases
as more observations are included in the fit. Thus, the vast majority of SNe
exhibit an increase in $\alpha_g$ and \tfl\ as $f_\mathrm{thresh}$ is
increased. Figure~\ref{fig:flux_frac} shows that the magnitude of this trend
is much larger for $\alpha_g$ than \trise, which makes sense. When there are
few SN detections, which is more likely when $f_\mathrm{thresh}$ is low, small
values of $\alpha$ fit the data well, as in Figure~\ref{fig:biggap_lc}.
Including more information about the rise, by increasing $f_\mathrm{thresh}$,
results in very-low values of $\alpha$ no longer being consistent with the
data. \tfl, on the other hand, is strongly constrained by the first epoch of
detection (see \S\ref{sec:redshift_correlations}). In this case the addition
of more observations will not lead to as dramatic an effect.

\subsection{The Importance of Pre-Explosion Observations}\label{sec:pre_explosion}

A unique, and important, component of our ZTF data set is the nightly
collection of multiple observations. \citet{Yao19} demonstrated that such an
observational sequence enables low-SNR detections of the SN prior to the
traditional $5\sigma$ discovery epoch (see \citealt{Masci19}), which can
provide critical constraints on \tfl. Many previous studies utilize filtered
observations that are obtained $\sim$1\,d, or more, after the epoch of
discovery (e.g., \citealt{Riess99a,Aldering00,Ganeshalingam10,Zheng17a}). To
demonstrate the importance of the ZTF sub-threshold detections, we re-fit the
model from \S\ref{sec:model} to each of our ZTF light curves \textit{after
removing all observations before and on the night the SN is first detected}
(i.e., SNR\,$\geq 5$, as defined in \citealt{Yao19}).

Following the removal of these observations, the MCMC chains converge (see
\S\ref{sec:model}) for only 10 SNe. This is understandable as the removal of
the ``baseline'' observations makes it very difficult to constrain $C_d$ and
$\beta_d$. The removal of these observations leads to dramatically different
estimates of the model parameters for these 10 SNe. Thus, we report the
results given the strong trends, though we caution that these results are
somewhat preliminary and should be confirmed with more detailed simulations.

With the baseline observations removed, the inferred value of \tfl\ increases
(i.e., \trise\ decreases) for all 10 SNe relative to the results from
\S\ref{sec:model}. The median difference of this shift is $\sim$3.5\,d. Using
the definition of agreement from \ref{sec:flux_cut}, i.e., overlap in the 90\%
credible regions, only 3 of the 10 SNe have estimates of \tfl\ that agree
after removing the non-detections. Removing the baseline observations also
decreases estimates of $\alpha$ (which agrees with the trend seen in
Figure~\ref{fig:model_parameters}), with only 5 of the 10 SNe having estimates
of $\alpha_g$ and $\alpha_r$ that agree. These trends suggest that
pre-explosion observations are critically needed to produce accurate estimates
of \trise\ (though, again, we caution that these results should be confirmed
with a larger sample of SNe).

\bibliography{/Users/adamamiller/Documents/tex_stuff/papers}
\bibliographystyle{aasjournal}

%% This command is needed to show the entire author+affiliation list when
%% the collaboration and author truncation commands are used.  It has to
%% go at the end of the manuscript.
%\allauthors

%% Include this line if you are using the \added, \replaced, \deleted
%% commands to see a summary list of all changes at the end of the article.
%\listofchanges

\end{document}